\section{0507}\label{sec:0507}
\begin{questions}
    \question 设有复数$x=a+b \bm i$和$y=c+d \bm i$,设计算法,只用3次乘法计算乘积$xy$
    \begin{solution}
        \[
            xy = (a+b \bm i) (c+d \bm i) = (ac-bd) + (ad+cb) \bm i
        \]
        \begin{enumerate}
            \item $A = ad                             $
            \item $B = bc                             $
            \item $C = (a+b)(c-d) = ac - ad + bc - bd $
        \end{enumerate}
        \begin{align*}
            ac-bd & = C + A - B \\
            ad+cb & = A + B
        \end{align*}
    \end{solution}

    \question 设$P$是一个$n$位十进制正整数。
    如果将$P$划分为$k$段,则可得到$k$个正整数,这$k$个正整数的乘积称为$P$的一个$k$乘积。
    \begin{parts}
        \part 求出$1234$的所有$2$乘积
        \begin{solution}
            \begin{align*}
                1 \times 234 & = 234  \\
                12 \times 34 & =  408 \\
                123 \times 4 & =  492
            \end{align*}
        \end{solution}

        \part 对于给定的$P$和$k$,求出$P$的最大$k$乘积的值
        \begin{solution}
            使用动态规划求解。
            设$Num(n, i)$是整数$n$的最低$i$位构成的数。
            设$Prod(n,k)$是整数$n$的最大$k$乘积。
            则\[
                Prod(n,k) = \begin{cases}
                    n                                                           & , k = 1 \\
                    \max_{i} \left\{ \frac{n-Num(n,i)}{10^i} \cdot Prod(Num(n,i),k-1) \right\} & , k > 1
                \end{cases}
            \]

        \end{solution}
    \end{parts}

    \question 分析在一般微机上如何计算二项式系数$C_n^k$
    \begin{solution}
    算法见\ref{0507:comb}

        亦可使用杨辉三角,$$C_{n}^{k}=C_{n-1}^{k-1}+C_{n-1}^{k}$$,
        这是一个$O(n^2)$的算法,可以最大限度避免溢出。
        但实验显示,算法 \ref{0507:comb} 在避免溢出方便仅比基于杨辉三角递归差了一丢丢。
    \end{solution}

    \begin{algorithm}[!htp]
        \caption{组合数}\label{0507:comb}
        \begin{algorithmic}[1]
            \Require {正整数$n,k$}
            \Ensure {$C_n^k$}
            \State $result \gets 1$
            \State $k \gets \min(k, n-k)$
            \For {$i \gets 1$ to $k$}
            \State $result \gets result \times (n-i+1)$
            \State $result \gets result \div i$
            \EndFor
            \State \Return $result$
        \end{algorithmic}
    \end{algorithm}

\end{questions}
