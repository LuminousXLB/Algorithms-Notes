\section{0413}\label{sec:0413}
\begin{questions}\label{qset:0413}

    \question 对于给定的二叉树,求其最小深度,即从根节点到最近的叶子的距离。

    \begin{solution}
        广度优先搜索最先到达的叶子节点有最小的深度。

        最差情况下遍历了整棵树,时间复杂度为$O(N)$,其中$N$为节点的数量。

        \textbf{伪代码见算法\ref{alg:0413:01}}
    \end{solution}

    \begin{algorithm}[!htp]
        \caption{求二叉树的最小深度} \label{alg:0413:01}
        \begin{algorithmic}[1]
            \Require {$root$(给定二叉树的根节点)}
            \Ensure {$mindepth$(二叉树中的最小深度)}
            \State $q \gets \Call{Queue.Init}{}$
            \State $\Call{Queue.Push}{q, \langle root, 0 \rangle}$
            \While {$\neg \Call{Queue.isEmpty}{q}$}
            \State $\langle current, depth \rangle \gets \Call{Queue.Pop}{q}$
            \If {$current.isLeaf$} \Comment{当前节点是叶子节点}
            \State \Return $depth$ \Comment{BFS最先到达的叶子节点有最小的深度}
            \EndIf

            \If {$current.left$} \Comment{左子树入队}
            \State $\Call{Queue.Push}{q, \langle current.left, depth+1 \rangle}$
            \EndIf

            \If {$current.right$} \Comment{右子树入队}
            \State $\Call{Queue.Push}{q, \langle current.right, depth+1 \rangle}$
            \EndIf
            \EndWhile
        \end{algorithmic}
    \end{algorithm}


    \question 设$G$是有向非循环图,其所有路径最多含$k$条边。
    设计线性时间算法,将所有顶点分为$k+1$组,每一组中任意两个点之间不存在路径。

    \begin{solution}
        DAG中一定存在入度为$0$的节点,且一定存在出度为$0$的节点,最长路径出现在入度为$0$的节点和出度为$0$的节点之间(否则可以沿两端点继续扩展路径,得到更长的路径)。

        按照拓扑排序的顺序,删除DAG中入度为$0$的节点后,上述最长路径的长度一定减小$1$。
        且在拓扑排序中同一批被移除的节点之间一定不存在路径(否则删除起点时,终点的入度一定不为0)。
        因此可按照拓扑排序中删除节点的批次将所有节点分为$k+1$组,且每一组中任意两个节点之间不存在路径。

        当图以出边邻接表给出时,算法的时间复杂度为$O(|V| + |E|)$。

        \textbf{伪代码见算法\ref{alg:0413:02}}

    \end{solution}

    \begin{algorithm}[!htp]
        \caption{拓扑分组} \label{alg:0413:02}
        \begin{algorithmic}[1]
            \Require {$G=(V,E)$(给定DAG的出边表)}
            \Ensure {$vtop\left[1 \dots |V|\right], bnd$($vtop\left[bnd[i] \dots bnd[i+1]\right]$中的节点不存在路径)}

            \State $vtop \gets \Call{Vector.Init}{}, bnd \gets \Call{Vector.Init}{}$

            \State $indeg \gets \Call{Array.Init}{$|V|, 0$}$  \Comment{构造入度表}
            \For {$(v,w)$ in $E$}
            \State $indeg[w] \gets indeg[w] + 1$
            \EndFor

            \For {$v$ in $V$ where $indeg[v] = 0$} \Comment{找到入度为$0$的节点}
            \State $\Call{Vector.PushBack}{vtop, v}$
            \EndFor

            \Statex \Comment{拓扑排序}
            \State $inf \gets 1, \Call{Vector.PushBack}{bnd, inf}$
            \State $sup \gets \Call{Vector.size}{vtop}, \Call{Vector.PushBack}{bnd, sup}$
            \While {$\Call{Vector.size}{vtop} < |V|$}
            \For{$idx \gets inf$ to $sup$} \Comment{对于每一个上一阶段找到的入度为$0$的节点}
            \For {$(v, w)$ in $E$ where $v = vtop[idx]$} \Comment{找到其后继节点}
            \State $indeg[w] \gets indeg[w] - 1$ \Comment{减后继节点的入度}
            \If{$indeg[w] = 0$} \Comment{并检查是否减到了$0$}
            \State $\Call{Vector.PushBack}{vtop, w}$
            \EndIf
            \EndFor
            \EndFor
            \State $inf \gets sup + 1$
            \State $sup \gets \Call{Vector.size}{vtop}, \Call{Vector.PushBack}{bnd, sup}$
            \EndWhile
        \end{algorithmic}
    \end{algorithm}

    \question 给定连通无向图$G$以及3条边$a,b,c$,在线性时间内判断$G$中是否存在一个包含$a$和$b$但不含$c$的闭链。

    \begin{solution}
        删除边$c$,在剩余的图中判断是否存在包含$a,b$的回路。

        在剩余的无向图中划分双连通分支(线性时间),若$a,b$在同一个双连通分支里,则存在一条包含$a,b$但不包含$c$的回路。
    \end{solution}

\end{questions}

