\documentclass[answers]{exam}
% \documentclass{article}

\usepackage[UTF8]{ctex}
\usepackage{fontspec}

\setCJKmainfont{SourceHanSerifCN-Regular}[
    Path = fonts/ ,
    Extension = .otf ,
    UprightFont = SourceHanSerifCN-Regular,
    BoldFont = SourceHanSerifCN-Bold,
]

\setCJKsansfont{SourceHanSansCN-Regular}[
    Path = fonts/ ,
    Extension = .otf ,
    UprightFont = SourceHanSansCN-Regular,
    BoldFont = SourceHanSansCN-Bold,
]

\usepackage{fullpage}
\usepackage{multicol}
\usepackage{color}
\usepackage{hyperref}
\usepackage{comment}

\usepackage{caption}
\usepackage{subcaption}

\usepackage{amsmath}
\usepackage{amssymb}
\usepackage{amsthm}
\usepackage{bm}
\usepackage{mathtools}

\newtheoremstyle{example}% name
{3pt}% Space above
{3pt}% Space below
{}% Body font
{}% Indent amount (empty = no indent, \parindent = para indent)
{\normalfont}% Thm head font
{}% Punctuation after thm head
{1em}% Space after thm head: " " = normal interword space;
% \newline = linebreak
{\textsf{\thmname{#1}\thmnumber{#2}\thmnote{#3}}}% Thm head spec (can be left empty, meaning `normal')

\theoremstyle{example}
\newtheorem{example}{例}

\renewenvironment{proof}{\noindent\textsf{证明}\quad}{\hfill $\square$\par}
\newtheorem{lemma}{引理}

\usepackage{algorithm}
\usepackage{algpseudocode}

\floatname{algorithm}{算法}
\renewcommand{\algorithmicrequire}{\textsf{输入:}}
\renewcommand{\algorithmicensure}{\textsf{输出:}}

\renewcommand{\solutiontitle}{}

\usepackage{tikz}
\usepackage{minted}


\title{算法与复杂性 \quad 作业二}

\author{516021910528 - SHEN Jiamin}
\date{2020年3月5日}

\begin{document}

\maketitle

\begin{questions}

    \question 数列$1,2,3,4,5,10,20,40,\dots$,该数列开始是等差数列,第5项以后为等比数列,证明任意一个正整数都可表示为这个数列中的不同数之和。

    \begin{solution}
        \begin{parts}
            \part{
                \begin{lemma}
                    对任意的$n \in \mathbb{N}^* , n < 2^k$,$n$可以表示为集合$\left\{ 2^i \mid 0 \leq i < k \right\}$中任意个不同数之和。
                    \label{BinaryCase}
                \end{lemma}

                \begin{proof}
                    定义集合$S_k$:
                    \begin{enumerate}
                        \item $\left\{ 2^i \mid 0 \leq i < k \right\} \subset S_k$
                        \item 若$a \in \mathbb{N}^*$可表示为集合$\left\{ 2^i \mid 0 \leq i < k \right\}$中任意多个不同数之和,则$a \in S_k$
                    \end{enumerate}
                    考察集合$S_k$:
                    \begin{enumerate}
                        \item 当$k=1$时,$S_1 = \left\{ 1 \right\}$
                        \item 当$k=2$时,$S_2 = \left\{ 1, 2, 3 \right\}$
                        \item {
                              假设当$k=n$时$S_n = \left\{ a \in \mathbb{N} \mid 1 \leq a < 2^{n} \right\}$ \\
                              则当$k=n+1$时,
                              \begin{align*}
                                  S_{n+1} & = S_n \cup \left\{ 2^n \right\} \cup \left\{ 2^n + a \mid a \in S_n \right\}                                                                             \\
                                          & = \left\{ a \in \mathbb{N} \mid 1 \leq a < 2^{n} \right\} \cup \left\{ 2^n \right\} \cup \left\{ a \in \mathbb{N} \mid 2^n + 1 \leq a < 2^{n+1} \right\} \\
                                          & = \left\{a \in \mathbb{N} \mid 1 \leq a < 2^{n+1} \right\}
                              \end{align*}
                              }
                    \end{enumerate}
                    归纳可知$S_k = \left\{a \in \mathbb{N} \mid 1 \leq a < 2^{k} \right\}$,即引理\ref{BinaryCase}成立。
                \end{proof}
            }

            \part{
                原数列通项公式可表示为$$ a_i =
                    \begin{cases}
                        i,              & 1 \leq i < 5 \\
                        5 \cdot 2^{i-5} & i \geq 5
                    \end{cases} $$

                对于任意正整数$ n \in \mathbb{N}^*$,
                \begin{enumerate}
                    \item {
                          若$n\leq 5$,易知$n \in \left\{ a_i \right\}$,即由数列中的单个数即可表示。
                          %   或$n\in \left\{5 \cdot 2^k \mid k \in \mathbb{N} \right\}$
                          }
                    \item {
                          由引理\ref{BinaryCase}和整数加法的线性性可推知,
                          $\forall n \in \mathbb{N}^* , 5 | n, n < 5 \cdot 2^k$,$n$可以表示为集合$\left\{ 5 \cdot 2^i \mid 0 \leq i < k \right\}$中任意个不同数之和。所以若$5|n$,则$n$可以表示为$\left\{ a_i \mid i \geq 5 \right\}$中任意个不同数之和。\label{timesOfFive}
                          }
                          % \item {
                          %       。由\ref{timesOfFive}可知。所以
                          %       }
                \end{enumerate}
                $$\mathbb{N}^* = \left\{ 5x+y \mid x \in \mathbb{N}, y\in \left\{1,2,3,4 \right\} \right\}$$
                其中$5x$可以表示为$\left\{ a_i \mid i \geq 5 \right\}$中任意个不同数之和;又$n \in \left\{ a_i \mid 1 \leq i < 5 \right\}$。归纳可知,任意正整数可以表示为$\left\{ a_i \right\}$中任意个不同数之和。
            }
        \end{parts}
    \end{solution}
    \begin{solution}
        \begin{proof}
            该数列通项公式可表示为$$ a_i =
                \begin{cases}
                    i,              & 1 \leq i < 5 \\
                    5 \cdot 2^{i-5} & i \geq 5
                \end{cases} $$

            对任意的$n \in \mathbb{N}^*$,存在唯一一组$(t,b,q)$满足$t\in \mathbb{N}, b\in \left\{0,1 \right\}, q \in \left\{ 1,2,3,4\right\}$,使得$$n = 5t+bq$$

            $t$有唯一的二进制表示,使$t = \sum_{k}{b_k \cdot 2^k}$,其中$b_k \in \left\{0,1\right\}$

            即\begin{align*}
                n = 5 \cdot \sum_{k}{b_k \cdot 2^k} + bq
                = \sum_{k}{b_k \cdot (5 \cdot 2^k)} + bq
            \end{align*}

            因为$b_k, b \in \left\{0,1\right\}$,不可能有重复的加数。又由题可知,
            \begin{align*}
                \left\{5 \cdot 2^k \right\} & = \left\{a_i \mid i \geq 5 \right\}     \\
                \left\{q \right\}           & = \left\{a_i \mid 1 \leq i < 5 \right\}
            \end{align*}
            所以任意的$n \in \mathbb{N}^*$可以表示为$\left\{ a_i \right\}$中任意多不同数之和。
        \end{proof}
    \end{solution}

    \newpage %%%%%%%%%%%%%%%%%%%%%%%%%%%%%%%%%%%%%%%%%%%%%%%%%%%%%%%%%%%%

    \question 广场上站着99个间谍,间谍与间谍之间的距离互不相等,每个间谍都盯着离自己最近的那个间谍看,证明总存在一个没被人盯着的间谍。

    \begin{solution}

        间谍与监视关系构成一张有向图,图中每个节点的出度为1。

        该图中不可能存在一个边数大于2的闭环(即$v_0 \rightarrow v_1 \rightarrow \dots \rightarrow v_n \rightarrow v_0 $)。
        若存在这样一个闭环,则一定存在闭环上的一条边$(v_i, v_{i+1})$是环上所有边中最短的,所以$v_i, v_{i+1}$必互相监视,与假设不符。
        所以该图中不可能存在一个边数大于2的闭环。

        若无法形成闭环,由于每个节点出度为1,其每个连通子图只能存在以下情况
        \begin{itemize}
            \item 两节点相互监视,且无其他节点监视这两个节点。
            \item 链形$v_0 \rightarrow v_1 \rightarrow \dots \rightarrow v_n$。
                  由于不可能存在闭环,要使$v_n$出度为1必有$v_n \rightarrow v_{n-1}$,此时必然存在$v_0$的入度为0。
            \item 星形(即$v_1 \rightarrow v_0, v_2 \rightarrow v_0,\dots $)。
                  由于$v_0$出度为1,只能与其中一个节点互相监视,周围其他节点的入度均为$0$。
        \end{itemize}

        当有99个间谍时,即使两两一对互相监视,也一定存在1个节点无法成对。
        因此一定存在星形或链形结构,使存在至少一个人没有被盯着。

    \end{solution}

    \newpage %%%%%%%%%%%%%%%%%%%%%%%%%%%%%%%%%%%%%%%%%%%%%%%%%%%%%%%%%%%%

    \question 有10个海盗抢得了100枚金币,每个海盗都能够很理智地判断自己的得失,他们决定这样分配金币:
    \begin{enumerate}
        \item 按照强壮与否排序,其中最强壮的人为10号,以此类推,最瘦小的人为1号。
        \item 先由10号提出分配方案,然后由所有人表决,当且仅当\textbf{等于或多于半数人}(包括自己)同意时,方案才算被通过,否则他将被扔入大海喂鲨鱼;
        \item 如果10号死了,将由9号提方案,其余的人表决,当且仅当\textbf{超过半数}(包括自己)同意时,方案才算通过,否则9号同样将被扔入大海喂鲨鱼;
        \item 往下以此类推……
    \end{enumerate}
    海盗们都很精明,他们首先会尽量保住自己的命,其次在保住命的前提下都想分到尽可能多的金币,而且他们也很希望自己的同伴喂鲨鱼。

    \begin{parts}
        \part 假如你是那个1号海盗,你将怎样分配,才能既保住命,又能分到最多的金币?最多能分到多少呢?
        \part 如果还是100枚金币,但海盗的数量是20,50,100,200,400又该怎么样呢?
    \end{parts}


    \begin{solution}
        设有$n$个海盗时,$n$号海盗提出的方案中,$j$号海盗分得的金币数量为$c_n(j)$。

        当仅剩两人(1、2号)时,无论2号提出怎样的方案,1号都会选择不同意从而把2号扔进大海并分得全部金币。
        所以,当剩余三人时(1、2、3号),无论3号提出怎样的方案,2号都会同意3号的方案以避免出现上述情况。
        因此,3号会提出一个$$
            c_3(j) = \begin{cases}
                100 & , j=3   \\
                0   & , j=1,2
            \end{cases}
        $$的方案,2、3号海盗会同意这个方案。

        在上述方案中,1、2号海盗都没有得到金币。
        因此,当由4号海盗提出方案时,只要他分给1号或2号海盗1个金币,即可赢得其支持。如$$
            c_4(j) = \begin{cases}
                99 & , j=4   \\
                1  & , j=2   \\
                0  & , j=1,3 \\
            \end{cases}
        $$

        当由5号海盗提出方案时,需要3人同意方可通过。
        代价最小的方案为在$c_4$的基础上多分给1、3号各一个金币。即$$
            c_5(j) = \begin{cases}
                98 & , j=5   \\
                1  & , j=1,3 \\
                0  & , j=2,4
            \end{cases}
        $$

        当由6号海盗提出方案时,需要3人同意即可通过。
        代价最小的方案为$c_5$基础上分给2、4号一个金币。如$$
            c_6(j) = \begin{cases}
                98 & , j=6     \\
                1  & , j=2,4   \\
                0  & , j=1,3,5
            \end{cases}
        $$

        \textsf{已知} \quad 当有$n$个海盗时,须有$
            \left\lceil \frac{n}{2} \right\rceil = \begin{cases}
                \frac{n+1}{2} & , n \text{为奇数} \\
                \frac{n}{2}   & , n \text{为偶数}
            \end{cases}
        $个海盗支持方可通过方案。

        \textsf{猜想} \quad 当$3 < n \leq 200$时, $n$号海盗提出如下方案可保住自己的命:$$
            c_n(j) = \begin{cases}
                101-\left\lceil \frac{n}{2} \right\rceil & , j=n                     \\
                1                                        & , j<n , j+n \text{为偶数} \\
                0                                        & , j<n , j+n \text{为奇数}
            \end{cases}
        $$

        \begin{proof}
            下面使用数学归纳法证明上述猜想。
            \begin{enumerate}
                \item 前已论述$n \leq 6$时的情况
                \item {
                      假设$n=k$时猜想成立
                      \begin{itemize}
                          \item {
                                若$k$为奇数,则$k$号海盗可分得$101-\frac{k+1}{2}$个金币。
                                且有$\frac{k+1}{2}-1$个奇数号海盗获得了$1$个金币,$\frac{k-1}{2}$个偶数号海盗没有获得金币。
                                }
                          \item {
                                若$k$为偶数,则$k$号海盗可分得$101-\frac{k}{2}$个金币。
                                且有$\frac{k}{2}-1$个偶数号海盗获得了$1$个金币,$\frac{k}{2}$个奇数号海盗没有获得金币。
                                }
                      \end{itemize}

                      则当$n=k+1$时
                      \begin{itemize}
                          \item {
                                若$k$为奇数,$k+1$为偶数。
                                $k+1$号海盗须得到另外$\frac{k-1}{2}$个海盗的支持方可通过方案。
                                因此,只需给$n=k$时没有分得金币的$\frac{k-1}{2}$个偶数号海盗分1个金币即可赢得他们的支持。
                                即$k+1$为偶数时,$$
                                    c_{k+1}(j) = \begin{cases}
                                        100 - \frac{k-1}{2} = 101 - \lceil \frac{k+1}{2} \rceil & , j=k+1                  \\
                                        1                                                       & , j<k+1, j \text{为偶数} \\
                                        0                                                       & , j<k+1, j \text{为奇数}
                                    \end{cases}
                                $$
                                }
                          \item {
                                若$k$为偶数,$k+1$为奇数。
                                $k+1$号海盗须得到另外$\frac{k}{2}$个海盗的支持方可通过方案。
                                因此,只需给$n=k$时没有分得金币的$\frac{k}{2}$个奇数号海盗分1个金币即可赢得他们的支持。
                                即$k+1$为奇数时,$$
                                    c_{k+1}(j) = \begin{cases}
                                        100 - \frac{k}{2} = 101 - \lceil \frac{k+1}{2} \rceil & , j=k+1                  \\
                                        1                                                     & , j<k+1, j \text{为奇数} \\
                                        0                                                     & , j<k+1, j \text{为偶数}
                                    \end{cases}
                                $$
                                }
                      \end{itemize}
                      }
            \end{enumerate}
            综合可知上述猜想成立。
        \end{proof}

        由此可知,
        \begin{itemize}
            \item 当$n=200$时,$
                      c_{200}(j) = \begin{cases}
                          1 & , j<200 , j \text{为偶数} \\
                          0 & , j<200 , j \text{为奇数}
                      \end{cases}
                  $

            \item 当$n=201$时,201号海盗必须获得除他自己以外的另外100名海盗的支持。
                  因此,他须分给1-200号海盗中100名奇数号海盗各1枚金币,而他本人无法分得金币。

            \item 当$n=202$时,202号海盗必须获得除他自己以外的另外100名海盗的支持。
                  因此,他须分给1-200号海盗中100名偶数号海盗各1枚金币,而他本人无法分得金币。

            \item 当$n=203$时,203号海盗必须获得除他自己以外的另外101名海盗的支持。
                  但他只有100枚金币,无法获得101名海盗的支持。所以他的方案不可能通过。

            \item 当$n=204$时,204号海盗必须获得除他自己以外的另外101名海盗的支持。
                  由上述可知,203号海盗一定会支持204号海盗以保护自己。
                  因此他只需在其他人中选100人,分给他们每人1个金币即可。

            \item 当$n=205$时,205号海盗必须获得除他自己、100个金币能收买的100个海盗以外,另外2名海盗的支持。
                  同理,当$n=206$、$n=207$时,方案都无法被通过。

            \item 当$n=208$时,208号海盗必须获得除他自己、100个金币能收买的100个海盗以外,另外3名海盗的支持。
                  205-207号海盗为保命也会支持208号海盗,因此208号海盗的方案会被通过。

        \end{itemize}

        \textsf{猜想} \quad 当$n>200$时,当且仅当$n = 200 + 2^k,k\in \mathbb{N}$时,存在可以通过的方案。

        \begin{proof}
            \begin{enumerate}
                \item 由上述论述可知,当$k=0,1,2,3$,$n=201,202,204,208$时,存在可以通过的方案,且本人均无法得到金币。
                \item {
                      设$k \in \mathbb{N}^*$。
                      \begin{itemize}
                          \item 若$n=200+2k$时,存在一个可以通过的方案,即$100 + k$个海盗支持了方案。
                                则当$n=201+2k$时,需要$101 + k$个海盗支持。
                                除他自己和能用金币收买的100个海盗(共101人)以外,剩余$k$个人均会选择拒绝方案从而将他扔进大海。
                          \item 若$n=200+2k$时,不存在一个可以通过的方案,即表示支持的海盗人数$\delta < 100 + k$。
                                则当$n=201+2k$时,表示支持的海盗人数为$\delta + 1 \leq 100 + k$,不足$101 + k$个。
                                方案仍然不会被通过。
                      \end{itemize}
                      所以当$n > 201$且$n$为奇数时,不存在能通过的方案。
                      }
                \item {
                      假设$n=200+2^k$时,存在可以通过的方案。
                      若$n = n' > 200+2^k$(由上述可知,$n'$定为偶数)时,也存在可以通过的方案,且$200+2^k < n < n'$时不存在可以通过的方案。
                      则$200+2^k$号至$n'-1$号共$n'-1 - (200+2^k)$名海盗都会支持$n'$号海盗。
                      因此若有$n'$名海盗时存在可以通过的方案,定有\begin{align*}
                          \left\lceil \frac{n'}{2} \right\rceil & = 100 + 1 + (n'-1 - (200+2^k)) = n' - 2^k - 100 \\
                          n'                                    & = 2n' - 2^{k+1} - 200                           \\
                          n'                                    & = 200 + 2^{k+1}
                      \end{align*}
                      }
            \end{enumerate}
            归纳可得,上述猜想成立。
        \end{proof}

        \textsf{答} \quad
        \begin{parts}
            \part{
                当$n=10$时,$
                    c_{10}(j) = \begin{cases}
                        96 & , j=n                    \\
                        1  & , j<10 , j \text{为偶数} \\
                        0  & , j<10 , j \text{为奇数}
                    \end{cases}
                $
            }
            \part{
                当$n=20$时,$
                    c_{20}(j) = \begin{cases}
                        91 & , j=n                    \\
                        1  & , j<20 , j \text{为偶数} \\
                        0  & , j<20 , j \text{为奇数}
                    \end{cases}
                $

                当$n=50$时,$
                    c_{50}(j) = \begin{cases}
                        76 & , j=n                    \\
                        1  & , j<50 , j \text{为偶数} \\
                        0  & , j<50 , j \text{为奇数}
                    \end{cases}
                $

                当$n=100$时,$
                    c_{100}(j) = \begin{cases}
                        51 & , j=n                     \\
                        1  & , j<100 , j \text{为偶数} \\
                        0  & , j<100 , j \text{为奇数}
                    \end{cases}
                $

                当$n=200$时,$
                    c_{200}(j) = \begin{cases}
                        1 & , j<200 , j \text{为偶数} \\
                        0 & , j<200 , j \text{为奇数}
                    \end{cases}
                $

                当$n=400$时,不存在自然数$k$使$200 + 2^k = 400$,因此400号海盗一定会被扔进大海。
            }
        \end{parts}
    \end{solution}

    \newpage %%%%%%%%%%%%%%%%%%%%%%%%%%%%%%%%%%%%%%%%%%%%%%%%%%%%%%%%%%%%

    \question 以下利用数学归纳法证明“所有的马颜色相同”错在哪儿
    \begin{enumerate}
        \item 只有一匹马时,命题成立
        \item {
              设有$n$匹马时命题成立。
              则当有$n+1$匹马$\{h_1,h_2,\dots,h_n,h_{n+1}\}$时, \\
              由归纳假设,$\{h_1,h_2,\dots,h_n\}$这$n$匹马颜色相同,
              $\{h_2,\dots,h_n,h_{n+1}\}$这$n$匹马的颜色相同, \\
              即$h_1$和$h_{n+1}$这两匹马与$\{h_2,\dots,h_n\}$颜色相同, \\
              所以$\{h_1,h_2,\dots,h_n,h_{n+1}\}$这$n+1$匹马的颜色是相同的
              }
    \end{enumerate}

    \begin{solution}
        考察由$n=1$推广至$n=2$时的情况:

        两匹马$\left\{ h_1, h_2 \right\}$中,
        由归纳假设$\left\{h_1\right\}$颜色相同,$\left\{h_2\right\}$颜色相同。
        但$\left\{h_1\right\} \cap \left\{h_2\right\} = \Phi$,
        所以不能推出$\left\{h_1,h_2\right\}$两匹马颜色相同。
        因此不能由此归纳“出所有的马颜色相同”。
    \end{solution}

\end{questions}









\end{document}