\section{0402}\label{sec:0402}
\begin{questions}

    \question 用分治法找到数组中的最大数和最小数,若数组规模为2的幂,证明需要的比较次数为$\frac{3}{2} n -2$
    \begin{solution}
        易判断初始条件$T(1) = 0, T(2) = 1$,
        且有递推关系
        \[ T(n) = T(2^m) = 2T(2^{m-1}) + 2 \Rightarrow T(2^m) + 2 = 2T(2^{m-1}) + 4 \]
        即
        \[ \frac{T(2^m) + 2}{T(2^{m-1}) + 2} = 2 \]

        所以当$m \ge 1$时,$T(2^m) + 2 = 3 \cdot 2^{m-1}$,
        可得$T(2^m) = \begin{cases}
                3 \cdot 2^{m-1} - 2 & , m \ge 1 \\
                1                   & , m = 0
            \end{cases}
        $

        即
        \[
            T(n) = \begin{cases}
                \frac{3}{2} n - 2 & , n \ge 2 \\
                1                 & , n = 1
            \end{cases}
        \]
    \end{solution}

    \question 对于任意给定的4个1-10之间的整数(可以相同),判断是否可以通过整数四则运算得到24
    \begin{solution}
        % TODO: complete this
        我们将运算中不具有交换律的情况算作两种运算,并规定$x_1 \leq x_2$,使两个数构成有序数对。
        则任一个有序数对的两个数之间可能有$6$种运算,即 \[
            x_1 + x_2 \quad x_1 \cdot x_2 \quad x_1 - x_2 \quad x_2 - x_1 \quad x_1 / x_2 \quad x_2 / x_1
        \]
        记$Q(a,b)$为$a$与$b$进行这$6$种计算得到的所有结果的集合,$|Q(a,b)| \le 6$。

        记$J(A,B,O,n)$为判断是否存在$A$中的某个数能和$B$中的某个数通过$O$中的运算直接得到$n$。
        对于每一种运算,使用类似于算法\ref{0330:Composite1}中的算法,
        首先扫描一遍较小的数组并在哈希表中记录下互补的值,然后扫描另一个数组看是否存在于哈希表中。
        这一部分需要进行$\min\left\{ |A|, |B| \right\}$次计算和$\max\left\{ |A|, |B| \right\}$次哈希表查找。
        所以$J(A,B)$的时间复杂度为$|O|(|A| + |B|)$。

        4个数的结合次序有两种,可表示为后缀表达式
        \begin{itemize}
            \item $(x_1, x_2, op_1, x_3, x_4, op_2, op_3)$:
                  4个数两两一组分为2组共有$\frac{\binom{4}{2} + \binom{2}{2}}{2}=3$种分组方式。
                  所以讨论全部结果需要执行3次\[
                      J\left( Q(x_1,x_2), Q(x_3, x_4), \left\{+, \times, -, \div, \hat{-} , \hat{\div} \right\}, 24 \right)
                  \]
                  总的运算次数为$3 \times 6 \times 12 = 216$
            \item $(x_1, x_2, x_3, x_4, op_1, op_2, op_3)$:
                  此时若$op_2$与$op_3$的运算顺序是可交换的,则与上一种情况相同。
                  \begin{itemize}
                      \item $op_2, op_3$同为加、减法(可通过增减括号后等价)
                      \item $op_2, op_3$同为乘、除法
                  \end{itemize}
                  因此,给定$op_2$后,$op_3$只有3种选择
                  排列方式共有$\binom{4}{2}\binom{2}{1} = 12$种。
                  \[
                      J\left(Q(Q(x_3, x_4), x_2), \left\{x_1\right\}, \left\{+, -, \hat{-}\right\}, 24\right)
                  \]
                  $Q(x_3, x_4)$的运算结果可由上一种情况的运算结果查表得到。
                  运算次数为$12 \times 6^2 \times 3 = 1296$次
        \end{itemize}

        总运算次数为$1512$。
    \end{solution}

    \question 有$n=2^k$位选手参加一项单循环比赛,即每位选手都要与其他$n-1$位选手比赛,且在$n-1$天内每人每天进行一场比赛
    \begin{parts}
        \part 设计算法以得到赛程安排
        \begin{solution}
            赛程安排可表示为表格$M$如下(以$k=2$为例):
            \begin{center}
                \begin{tabular}{c|cccc}
                      & 1        & 2        & 3        & 4        \\
                    \hline
                    1 & $\times$ & 0        & 1        & 2        \\
                    2 & $-$      & $\times$ & 2        & 0        \\
                    3 & $-$      & $-$      & $\times$ & 1        \\
                    4 & $-$      & $-$      & $-$      & $\times$ \\
                \end{tabular}
            \end{center}
            上表中,$m_{1,2} = 0$表示1号选手和2号选手在第0天进行一场比赛。
            由于自己不能和自己比赛,所以对角线上的值是无意义的。
            同时表格中的值应关于对角线对称,所以只需考虑对角线一侧的值(这里只考虑上方)。
            因此只要以某种算法,使用$[0, n-1)$中的整数填写上表,使每行、每列都不存在重复的数即可。

            填写表格方法为(算法\ref{0402:FillTable}):
            \begin{itemize}
                \item 第一个数:\[ m_{1,2} = 0 \]
                \item 第一行其他的数: \[ \forall j : 2 < j \le n, m_{1,j} = \left[ 1 + m_{1,j-1}  \bmod (n-1)  \right] \]
                \item 表中其他的数: \[ \forall i,j : i > 1, j > i, m_{i,j} = \left[ 1 + m_{i-1,j-1}  \bmod (n-1)  \right] \]
            \end{itemize}

            由于第一行中只有$n-1$个数,所以恰能使用$[0, n-1)$中所有的数填满且不重复。
            由于每一列中最多只有$n-1$个数,且有循环递增关系,所以一定能使用$[0, n-1)$中的数填满而不产生重复。
            由于第一行从左到右有循环递增关系,第二行的值为第一行对应的值$+1$,所以行中的递增关系保持,即每一行中也没有重复的数。
            因此由此方法填写的表格满足要求。

            该表格中安排了$\frac{n^2 - n}{2} = 2^{k-1}(n - 1)$,满足每天$k$场比赛,$n-1$天恰好比完。
            算法的时间复杂度为$O(n^2)$
        \end{solution}

        \begin{algorithm}[!htp]
            \caption{比赛安排}\label{0402:FillTable}
            \begin{algorithmic}[1]
                \Require {$n$(参加比赛的人数)}
                \Ensure {$M[1\dots n][1\dots n]$(比赛的安排,$i$号选手与$j$号选手在第$M[i][j]$天比赛)}
                \State $M[1][2] \gets 0$
                \For {$j \gets 3$ to $n$}
                \State $M[1][j] \gets M[1][j-1] \bmod (n-1)$
                \EndFor
                \For {$i \gets 2$ to $n$}
                \For {$j \gets i+1$ to $n$}
                \State $M[i][j] \gets M[i-1][j] \bmod (n-1)$
                \EndFor
                \EndFor
            \end{algorithmic}
        \end{algorithm}

        \part 若比赛结果存放在一矩阵中,针对该矩阵设计$O(n \log n)$的算法,为各个选手赋予次序$P_i$,
        使得$P_1$打败$P_2$,$P_2$打败$P_3$,依此类推。
        \begin{solution}
            对于任意两个选手$i,j$,定义若$i$赢了$j$则$i<j$。
            由于每两个人之间都有一场比赛,所以任意两个人之间都可以按上述定义比较大小,该大小关系可以通过查询比赛结果的数组得到。
            使用一种基于比较的排序算法(如快速排序),即可在$O(n\log n)$的时间内排除顺序。
        \end{solution}
    \end{parts}


    \question 数组$A$中包含$n$个互不相同的整数,用$O(n)$的时间找出$A$中最大的$i$个数,并按从大到小的次序输出($i \leq n^{1/2}$)
    \begin{solution}
        首先使用自底向上的方法建最大堆,然后出堆$i$次。

        \begin{itemize}
            \item 自底向上建堆过程中,每一步比较的次数最多是该节点高度的2倍。
                  对于高度为$i$的节点,记其所有子节点的高度和为$H(i)$,则有
                  \begin{align*}
                      H(i) = 2H(i-1) + 1 \xRightarrow{H(0) = 0} H(i) = 2^{i+1} - i - 2
                  \end{align*}
                  而$n$个节点的最大高度为$\log n$,所以这一部分的时间复杂度为\[
                      T_1(n) = 4n - 2 \lceil \log n \rceil - 4
                  \]
            \item 出堆$i$次所需的时间$ T_2(i) \le i \log n $。
        \end{itemize}
        所以整体的时间复杂度\begin{align*}
            T(n, i) = T_1(n) + T_2(i) & \le (4n - 2\lceil \log n \rceil - 4) + i \log n             \\
                                      & \le 4n - 2\lceil \log n \rceil - 4 + \sqrt{n} \log n        \\
                                      & \le 4n - 2\lceil \log n \rceil - 4 + \sqrt{n} \cdot n^{1/2} \\
                                      & = 5n - 2\lceil \log n \rceil - 4 = O(n)
        \end{align*}

    \end{solution}

\end{questions}
