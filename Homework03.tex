\section{0309}

\begin{questions}

    \question $k$为正常数,证明$\log{n} = o(n^k)$
    \begin{solution}
        \begin{proof}
            只须证\[
                \lim_{n \rightarrow + \infty} \frac{ \log_m n }{ n^k } = 0
            \]
            即证\[
                \lim_{n \rightarrow + \infty} \frac{ \ln n }{ n^k \ln m } = 0
            \]

            应用洛必达法则,有\begin{align*}
                \lim_{n \rightarrow + \infty} \frac{ \ln n }{ n^k \ln m }
                 & = \lim_{n \rightarrow + \infty} \frac{ 1/ n }{ k n^{k-1} \ln m } \\
                 & = \lim_{n \rightarrow + \infty} \frac{ 1 }{ k n^{k} \ln m } = 0
            \end{align*}

            所以\[
                \log{n} = o(n^k)
            \]
        \end{proof}
    \end{solution}

    \begin{solution}
        \begin{proof}
            \begin{enumerate}
                \item 当$0 < m < 1$时,$\forall c > 0, n > 1, k > 0$ $$
                          \log_m{n} < 0 < c n^k
                      $$
                \item 当$m > 1$时,令\[
                          f(x) = \frac{\log_m{x}}{ x^k }
                          \quad \Longrightarrow \quad
                          \ln{m} \cdot f'(x) = \frac{1 - k \ln{x}}{ x^{k+1}  }
                      \]
                      由$k > 0$,$\ln m > 0$可知,当$x > e^{1/k}$时,$f'(x) < 0$,$f(x)$单调递减

                      $\forall c > 0, \exists N > \max\left\{ m^c, e^{1/k} \right\}  > 0$,
                      使得当$n > N$时 $$
                          f(n) < f(m^c) =\frac{\log_m{m^c}}{m^{ck}} = \frac{c}{m^{ck}}
                      $$
                      由$ck>0$,$m>1$可知,$m^{ck} > 1$
                      \begin{align*}
                          c \cdot m^{ck} & > c                                                      \\
                          c              & > \frac{c}{m^{ck}} = f(N) > f(n) = \frac{\log_m{n}}{n^k} \\
                          \log_m{n}      & < c n^k
                      \end{align*}
                      所以$$\log_m{n} = o(n^k)$$
            \end{enumerate}
        \end{proof}
    \end{solution}

    % %%%%%%%%%%%%%%%%%%%%%%%%%%%%%%%%%%%%%%%%%%%%%%%%

    \question 寻找单调递增函数$f(n)$和$g(n)$,使得$f(n)=O(g(n))$和$g(n)=O(f(n))$都不成立

    \begin{solution}
        即寻找单调递增函数$f(n)$和$g(n)$,使得$\forall c > 0, \forall N > 0$ \begin{align*}
            \exists n_1 > N, f(n_1) > c g(n_1) , \quad
            \exists n_2 > N, g(n_2) > c f(n_2)
        \end{align*}

        令$k = \left\lfloor \frac{\left\lfloor \log_2{n} \right\rfloor}{2} \right\rfloor \in\mathbb{N}$
        $$
            f(n) = \begin{cases}
                2^n          & , 2^{2k} \leq n < 2^{2k+1}  \\
                2^{2^{2k+1}} & , 2^{2k+1} \leq n < 2^{k+2}
            \end{cases}
        $$
        $$
            g(n) = \begin{cases}
                2^{2^{2k}} & , 2^{2k} \leq n < 2^{k+1}    \\
                2^n        & , 2^{2k+1} \leq n < 2^{2k+2}
            \end{cases}
        $$
        则,当$2^{2k} \leq n < 2^{2k+1}$时,
        \begin{equation} \label{fog}
            \frac{f(n)}{g(n)} = \frac{2^n}{2^{2^{2k}}} = 2^{(n-2^{2k})}
        \end{equation}
        当$2^{2k+1} \leq n < 2^{2k+2}$时,
        \begin{equation} \label{gof}
            \frac{g(n)}{f(n)} = \frac{2^n}{2^{2^{2k+1}}} = 2^{(n-2^{2k+1})}
        \end{equation}

        $\forall c > 0, N > 0$,一定存在一个足够大的$m > 0$,使得$c < 2^{2^{2m}}$且$N < 2^{2m}$
        \begin{itemize}
            \item 由(\ref{fog})式可知,当$n = 2^{2m+3}-1 > N$时,
                  $2^{2m+2} < n < 2^{2m + 3}$,
                  \begin{align*}
                      f(n) & = 2^{(2^{2m+3}-1-2^{2m+2})} g(n) = 2^{(2^{2m+2}-1)} g(n) \\
                           & > 2^{2^{2m}} g(n) > c g(n)
                  \end{align*}
            \item 由(\ref{gof})式可知,当$n = 2^{2m+4}-1 > N$时,
                  $2^{2m + 3} < n < 2^{2m + 4}$,
                  \begin{align*}
                      g(n) & = 2^{(2^{2m+4}-1-2^{2m+3})} f(n) = 2^{(2^{2m+3}-1)} f(n) \\
                           & > 2^{2^{2m}} f(n) > c f(n)
                  \end{align*}
        \end{itemize}

        综合可知,对于上述$f(n),g(n)$,$f(n)=O(g(n))$和$g(n)=O(f(n))$都不成立
    \end{solution}


    \begin{solution}
        官方解答:
        \begin{enumerate}
            \item \[
                      f(n) = \begin{cases}
                          n!     & , n\text{为奇数} \\
                          (n+1)! & , n\text{为偶数}
                      \end{cases}, \quad
                      g(n) = \begin{cases}
                          (n+1)! & , n\text{为奇数} \\
                          n!     & , n\text{为偶数}
                      \end{cases}
                  \]
            \item \[
                      f(n) = \begin{cases}
                          f(n-1) + 1          & , n\text{为奇数} \\
                          f(n-1) + (g(n-1))^2 & , n\text{为偶数}
                      \end{cases}, \quad
                      g(n) = \begin{cases}
                          g(n-1) + (f(n-1))^2 & , n\text{为奇数} \\
                          g(n-1) + 1          & , n\text{为偶数}
                      \end{cases}
                  \]
        \end{enumerate}


    \end{solution}


    % %%%%%%%%%%%%%%%%%%%%%%%%%%%%%%%%%%%%%%%%%%%%%%%%

    \question 假设解决同一个问题的两个算法A1和A2的时间复杂性分别为$O(n^3)$和$O(n)$,
    如果为这两个算法分别编写程序并在同样的环境下运行,
    算法A2的程序一定比算法A1的程序运行得快吗?为什么?

    \begin{solution}
        不一定。
        程序的运行速度除了取决于时间复杂度,还取决于输入的规模以及程序的质量。
        在时间复杂度中,除了主项,还包括被忽略的常数以及其他项可能影响实际的运行时间。

        如算法A1的程序可能运行的时间为$n^3$,而算法A2的程序可能运行时间为$256n$。
        此时,若输入规模$n < 16$,则算法A1比算法A2运行得更快。
    \end{solution}

    % %%%%%%%%%%%%%%%%%%%%%%%%%%%%%%%%%%%%%%%%%%%%%%%%

    \question 考虑以下冒泡排序算法。
    \begin{algorithm}
        \caption{BubbleSort}
        \begin{algorithmic}[1]
            \Require $n$个元素的数组$A[1 \dots n]$
            \Ensure 按非降序排列的数组$A[1 \dots n]$
            \State $i \gets 1, sorted \gets \mathsf{false}$
            \While {$i \leq n-1 \wedge sorted \neq \mathsf{true}$}
            \State $sorted \gets \mathsf{true}$
            \For {$j \gets n$ downto $i+1$}
            \If {A[j] < A[j-1]}
            \State \Call{swap}{$A[j],A[j-1]$}
            \State $sorted \gets \mathsf{false}$
            \EndIf
            \EndFor
            \State  $i \gets i+1$
            \EndWhile
        \end{algorithmic}
    \end{algorithm}
    \begin{parts}
        \part 元素比较的最少次数是多少?何时达到最小值?
        \begin{solution}
            当数组本身为非降序时可达到最小值。
            此时外层循环只执行1次,内层循环只执行$n-1$次。
            所以元素比较次数最少为$n-1$次。
        \end{solution}
        \part 元素比较的最多次数是多少?何时达到最大值?
        \begin{solution}
            当数组本身为严格升序时可达到最大值。
            此时外层循环执行$n-1$次,内层循环共执行次数为$$
                \sum_{i=1}^{n-1}{(n-i)} = \frac{n(n-1)}{2}
            $$
            所以元素比较的最多次数为$\frac{n(n-1)}{2}$次。
        \end{solution}
        \part 可以用$O$,$\Omega$,$\Theta$表示算法的运行时间吗?
        \begin{solution}
            \textbf{
                \color{red} CONFUSED
            }
            \begin{itemize}
                \item 最好情况:$$n-1 = O(n) = \Omega(n) = \Theta(n)$$
                \item 最坏情况:$$\frac{1}{2}(n^2-n) = O(n^2) = \Omega(n^2) = \Theta(n^2)$$
            \end{itemize}
        \end{solution}
    \end{parts}

\end{questions}
