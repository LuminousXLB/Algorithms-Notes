\section{0302}

\begin{questions}

    \question 求解一元二次方程$ax^2+bx+c=0$

    \begin{solution}
        首先使用根的判别式判断方程根的情况,然后使用求根公式求解。

        {
        \color{red}
        有观点认为应当先判断$a$是否为零,但我认为一元二次方程已经隐含了$a \ne 0$的条件,
        无需额外判断。\url{https://en.wikipedia.org/wiki/Quadratic_equation}
        }

        \textbf{伪代码见算法\ref{0302:equation}}
    \end{solution}

    \begin{algorithm}[!htp]
        \caption{求实系数一元二次方程的根}\label{0302:equation}
        \begin{algorithmic}[1]
            \Require 实系数一元二次方程的系数$a,b,c$
            \Ensure 方程的根$x_1,x_2$
            \State $\Delta \gets b^2-4ac$ \Comment{根的判别式}
            \If {$\Delta > 0$} \Comment{方程有两个不同的实数根}
            \State $x_1 \gets \frac{-b + \sqrt{\Delta}}{2a}$
            \State $x_2 \gets \frac{-b - \sqrt{\Delta}}{2a}$
            \ElsIf {$\Delta = 0$} \Comment{方程有两个相等的实数根}
            \State $x_{1,2} \gets -\frac{b}{2a}$
            \ElsIf {$\Delta < 0$} \Comment{方程有两个不同的复数根}
            \State $x_1 \gets \frac{-b + \bm{i}\sqrt{-\Delta}}{2a}$
            \Comment{其中$\bm{i}^2=-1$}
            \State $x_2 \gets \frac{-b - \bm{i}\sqrt{-\Delta}}{2a}$
            \EndIf
            \State \Return $x_1, x_2$
        \end{algorithmic}
    \end{algorithm}

    \question 有一堆棋子,A和B两人轮流从中拿1-3个,A第一个拿,那么A如何确保自己不拿到最后一个棋子

    \begin{solution}

        若每次B拿$b_i$个棋子($b_i \in [1,3], b_i \in \mathbb{N}$)之后,
        A都拿$a_{i+1} = 4 - b_{i}$个棋子,使$b_i + a_{i+1} = 4$,则最后剩余棋子的个数可以认为与B的决策无关。
        要保证A不拿到最后一个棋子,则A最后一次拿取之后必须只剩1个棋子。
        否则,B可以拿取若干棋子使剩余棋子个数为1,A将不得不拿到最后一个棋子。
        由此可知,只要A在每一步拿取棋子时,都尽力保证剩余棋子的个数$N$满足$N \equiv 1 \bmod 4$即可。

        注:A的策略只依赖于其决策时的状态。既不依赖于历史状态,也不依赖于B的决策细节。

        \textbf{伪代码见算法\ref{0302:bash-game}}

    \end{solution}

    \begin{algorithm}[!htp]
        \caption{巴什博弈}\label{0302:bash-game}
        \begin{algorithmic}[1]
            \Require 轮到A拿棋子时,剩余棋子的个数$n$
            \Ensure A本次拿取棋子的个数
            \State $take \gets [(n-1) \bmod 4]$
            \If {$take > 0$}
            \State \Return $take$ \Comment{A此次拿$take$个棋子,可满足前述条件}
            \Else
            \State \Return $1$ \Comment{A不可能通过此次拿取保证前述条件,可随意拿若干棋子}
            \EndIf
        \end{algorithmic}
    \end{algorithm}

\end{questions}
