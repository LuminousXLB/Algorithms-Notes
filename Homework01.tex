\documentclass{article}

\usepackage[UTF8]{ctex}
\usepackage{fontspec}

\setCJKmainfont{SourceHanSerifCN-Regular}[
    Path = fonts/ ,
    Extension = .otf ,
    UprightFont = SourceHanSerifCN-Regular,
    BoldFont = SourceHanSerifCN-Bold,
]

\setCJKsansfont{SourceHanSansCN-Regular}[
    Path = fonts/ ,
    Extension = .otf ,
    UprightFont = SourceHanSansCN-Regular,
    BoldFont = SourceHanSansCN-Bold,
]

\usepackage{fullpage}
\usepackage{multicol}
\usepackage{color}
\usepackage{hyperref}
\usepackage{comment}

\usepackage{caption}
\usepackage{subcaption}

\usepackage{amsmath}
\usepackage{amssymb}
\usepackage{amsthm}
\usepackage{bm}
\usepackage{mathtools}

\newtheoremstyle{example}% name
{3pt}% Space above
{3pt}% Space below
{}% Body font
{}% Indent amount (empty = no indent, \parindent = para indent)
{\normalfont}% Thm head font
{}% Punctuation after thm head
{1em}% Space after thm head: " " = normal interword space;
% \newline = linebreak
{\textsf{\thmname{#1}\thmnumber{#2}\thmnote{#3}}}% Thm head spec (can be left empty, meaning `normal')

\theoremstyle{example}
\newtheorem{example}{例}

\renewenvironment{proof}{\noindent\textsf{证明}\quad}{\hfill $\square$\par}
\newtheorem{lemma}{引理}

\usepackage{algorithm}
\usepackage{algpseudocode}

\floatname{algorithm}{算法}
\renewcommand{\algorithmicrequire}{\textsf{输入:}}
\renewcommand{\algorithmicensure}{\textsf{输出:}}

\renewcommand{\solutiontitle}{}

\usepackage{tikz}
\usepackage{minted}


\title{算法与复杂性 \quad 作业一}

\author{516021910528 - SHEN Jiamin}
\date{2020年3月2日}

\begin{document}

\maketitle

用伪代码写出以下两个问题的算法,建议先用两三句话写出基本思想,再仿照课上最大公约数的例子写出伪代码

\section{求解一元二次方程$ax^2+bx+c=0$}

首先使用根的判别式判断方程根的情况,然后使用求根公式求解。

\begin{algorithm}
    \caption{求实系数一元二次方程的根}\label{QuadraticEquation}
    \begin{algorithmic}[1]
        \Require 实系数一元二次方程的系数$a,b,c$
        \Ensure 方程的根$x_1,x_2$
        \State $\Delta \gets b^2-4ac$ \Comment{根的判别式}
        \If {$\Delta > 0$} \Comment{方程有两个不同的实数根}
        \State $x_1 \gets \frac{-b + \sqrt{\Delta}}{2a}$
        \State $x_2 \gets \frac{-b - \sqrt{\Delta}}{2a}$
        \ElsIf {$\Delta = 0$} \Comment{方程有两个相等的实数根}
        \State $x_{1,2} \gets -\frac{b}{2a}$
        \ElsIf {$\Delta < 0$} \Comment{方程有两个不同的复数根}
        \State $x_1 \gets \frac{-b + \bm{i}\sqrt{-\Delta}}{2a}$
        \Comment{其中$\bm{i}^2=-1$}
        \State $x_2 \gets \frac{-b - \bm{i}\sqrt{-\Delta}}{2a}$
        \EndIf
        \State \Return $x_1, x_2$
    \end{algorithmic}
\end{algorithm}

\newpage

\section{有一堆棋子,A和B两人轮流从中拿1-3个,A第一个拿,那么A如何确保自己不拿到最后一个棋子}

若每次B拿$b_i$个棋子($b_i \in [1,3], b_i \in \mathbb{N}$)之后,A都拿$a_{i+1} = 4 - b_{i}$个棋子,使$b_i + a_{i+1} = 4$,则最后剩余棋子的个数可以认为与B的决策无关。

要保证A不拿到最后一个棋子,则A最后一次拿取之后必须只剩1个棋子。
否则,B可以拿取若干棋子使剩余棋子个数为1,A将不得不拿到最后一个棋子。

由此可知,只要A在每一步拿取棋子时,都尽力保证剩余棋子的个数$N$满足$N \equiv 1 \bmod 4$即可。

注:A的策略只依赖于其决策时的状态。既不依赖于历史状态,也不依赖于B的决策细节。

\begin{algorithm}
    \caption{巴什博弈}\label{VariedBashGame}
    \begin{algorithmic}[1]
        \Require 轮到A拿棋子时,剩余棋子的个数$n$
        \Ensure A本次拿取棋子的个数
        \State $take \gets [(n-1) \bmod 4]$
        \If {$take > 0$}
        \State \Return $take$ \Comment{A此次拿$take$个棋子,可满足前述条件}
        \Else
        \State \Return $1$ \Comment{A不可能通过此次拿取保证前述条件,可随意拿若干棋子}
        \EndIf
    \end{algorithmic}
\end{algorithm}


\end{document}