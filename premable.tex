\usepackage[UTF8]{ctex}
\usepackage{fontspec}

\setCJKmainfont{SourceHanSerifCN-Regular}[
    Path = fonts/ ,
    Extension = .otf ,
    UprightFont = SourceHanSerifCN-Regular,
    BoldFont = SourceHanSerifCN-Bold,
]

\setCJKsansfont{SourceHanSansCN-Regular}[
    Path = fonts/ ,
    Extension = .otf ,
    UprightFont = SourceHanSansCN-Regular,
    BoldFont = SourceHanSansCN-Bold,
]

\usepackage{fullpage}
\usepackage{multicol}
\usepackage{color}
\usepackage{hyperref}
\usepackage{comment}

\usepackage{caption}
\usepackage{subcaption}

\usepackage{amsmath}
\usepackage{amssymb}
\usepackage{amsthm}
\usepackage{bm}
\usepackage{mathtools}

\newtheoremstyle{example}% name
{3pt}% Space above
{3pt}% Space below
{}% Body font
{}% Indent amount (empty = no indent, \parindent = para indent)
{\normalfont}% Thm head font
{}% Punctuation after thm head
{1em}% Space after thm head: " " = normal interword space;
% \newline = linebreak
{\textsf{\thmname{#1}\thmnumber{#2}\thmnote{#3}}}% Thm head spec (can be left empty, meaning `normal')

\theoremstyle{example}
\newtheorem{example}{例}

\renewenvironment{proof}{\noindent\textsf{证明}\quad}{\hfill $\square$\par}
\newtheorem{lemma}{引理}

\usepackage{algorithm}
\usepackage{algpseudocode}

\floatname{algorithm}{算法}
\renewcommand{\algorithmicrequire}{\textsf{输入:}}
\renewcommand{\algorithmicensure}{\textsf{输出:}}

\renewcommand{\solutiontitle}{}

\usepackage{tikz}
\usepackage{minted}
