
\section{0330}\label{sec:0330}
\begin{questions}

    \question 有$n$个数存放在两个有序数组中,如何找到这$n$个数的第$k$小的数

    \begin{solution}
        比较两个数组的头部,将最小的一个提取出来。
        重复该步骤$k$次,第$k$个数即为第$k$小的数。
        该算法的时间复杂度为$O(k)$

        \begin{algorithm}[H]
            \caption{归并取第$k$小的数}\label{0330:MeargeSmallestK}
            \begin{algorithmic}[1]
                \Require{数组$A[1 \dots p],B[1 \dots q]$;目标$k$} \Comment{$p+q = n \geq k$}
                \Ensure{$A,B$中第$k$小的数$m$}
                \State $cnt \gets 0, i \gets 1, j \gets 1$
                \While {$cnt < k \wedge i \leq p \wedge j \leq q$}
                \If {$A[i] < B[j]$}
                \State $m \gets A[i], i \gets i + 1$
                \Else
                \State $m \gets B[j], j \gets j + 1$
                \EndIf
                \State $cnt \gets cnt + 1$
                \EndWhile

                \While {$cnt < k \wedge i < p$}
                \State $m \gets A[i], i \gets i + 1$
                \State $cnt \gets cnt + 1$
                \EndWhile

                \While {$cnt < k \wedge j < q$}
                \State $m \gets B[j], j \gets j + 1$
                \State $cnt \gets cnt + 1$
                \EndWhile
            \end{algorithmic}
        \end{algorithm}
    \end{solution}

    \begin{algorithm}[!ht]
        \caption{计数KMP}\label{0330:CountableKMP}
        \begin{algorithmic}[1]
            \Require $Text[1 \dots n]$(被查找字符串), $Pattern[1 \dots m]$(查找目标)
            \Ensure $cnt$($Pattern$在$Text$中出现的次数)
            \Procedure{StringMatch}{$Text,n,Pattern,m$}
            \State $next \gets \Call{ComputeNext}{Pattern, m}$
            \State $i \gets 1, j \gets 1$
            \State $ cnt \gets 0$
            \While {$i \leq n$}
            \If {$Pattern[j]=Text[i]$}
            \State $j \gets j+1,i \gets i+1$
            \If {$j=m+1$}
            \State $j \gets next[j] + 1$
            \State $cnt \gets cnt + 1$
            \EndIf
            \Else
            \State $j \gets next[j]+1$
            \If {$j=0$}
            \State $j \gets 1, i \gets i+1$
            \EndIf
            \EndIf
            \EndWhile
            \State \Return $cnt$
            \EndProcedure
            \Statex
            \Procedure{ComputeNext}{$Pattern, m$}
            \State $next[1 \dots m+1]$
            \State $next[1] \gets -1, next[2] \gets 0$
            \For {$i \gets 3$ to $m+1$}
            \State $j \gets next[i-1] + 1$
            \While {$j > 0 \wedge Pattern[i-1] \neq Pattern [j]$}
            \State $j \gets next[j] + 1$
            \EndWhile
            \State $next[j] \gets j$
            \EndFor
            \State \Return $next$
            \EndProcedure
        \end{algorithmic}
    \end{algorithm}

    \question 如何修改KMP算法,使之能够获得字符串$B$在字符串$A$中出现的次数

    \begin{solution}
        给定原有算法,已经能找到一个$s$,使得$\forall 1 \leq i \leq m, A[s+i-1] = B[i]$。

        如果存在一个$s' = s + \delta, 1 \leq \delta < m$,使得$\forall i \in [1,m], A[s'+i-1] = B[i]$
        (即$A$中有两个重叠的$B$,重叠长度为$m - \delta$),
        那么必然有即$\forall 1 \leq i \leq m-\delta, B[i] = B[i+\delta]$
        (即$B$长度为$m-\delta$的前后缀是相等的)。

        \textbf{伪代码见算法\ref{0330:CountableKMP}}
    \end{solution}

    \question 设计高效算法求序列$T$和$P$的最长公共子序列和最短公共超序列的长度。
    \begin{parts}
        \part 最长公共子序列(LCS)$L$定义为$T$和$P$的共同子序列中最长的一个
        \begin{solution}
            记$T[:i]$为序列$T$的前$i$个元素组成的序列。
            记$l_{LCS}(T,P)$为$T$和$P$的最长公共子序列的长度。
            不妨设序列$T$和$P$的长度分别为$m,n$。

            则有归纳关系\[
                l_{LCS}(T[:i],P[:j]) = \begin{cases}
                    l_{LCS}(T[:i-1],P[:j-1]) + 1                                       & , T[i] = P[j]   \\
                    \max\left\{l_{LCS}(T[:i],P[:j-1]), l_{LCS}(T[:i-1],P[:j]) \right\} & , T[i] \ne P[j] \\
                \end{cases}
            \]

            可用动态规划的方法解决此问题。
            该算法的时间复杂度为$O(|T \times P|)$,空间复杂度为$O(\min\left\{|T|, |P|\right\})$。


            \begin{algorithm}[H]
                \caption{最长公共子序列的长度}\label{0330:LCS}
                \begin{algorithmic}[1]
                    \Require $P[1 \dots m], T[1 \dots n]$(两个序列) \Comment{不妨设$m \ge n$}
                    \Ensure $l$($T,P$的最长公共子序列的长度)
                    \State $dp[0 \dots 1][0 \dots n] \gets \left\{ 0 \right\}$
                    \For {$i \gets 1$ to $m$}
                    \For {$j \gets 1$ to $n$}
                    \If{$T[i] = P[j]$}
                    \State $dp[i \bmod 2][j] \gets dp[(i-1) \bmod 2][j-1] + 1$
                    \ElsIf {$dp[i \bmod 2][j-1] \ge dp[(i-1) \bmod 2][j]$}
                    \State $dp[i \bmod 2][j] \gets dp[i \bmod 2][j-1]$
                    \Else
                    \State $dp[i \bmod 2][j] \gets dp[(i-1) \bmod 2][j]$
                    \EndIf
                    \EndFor
                    \EndFor
                    \State \Return $dp[m \bmod 2][n]$
                \end{algorithmic}
            \end{algorithm}
        \end{solution}

        \part 最短公共超序列(SCS)$S$定义为所有以$T$和$P$为子序列的序列中最短的一个
        \begin{solution}
            $T+P$一定是$T$和$P$的一个公共超序列,所以$|S| \le |T| + |P|$。

            当$T,P$存在一个公共子序列$Q$时,只需从$P$中移除一个$Q$,
            然后将$P$中其他元素按照与$T$中属于$Q$的元素相对顺序不变的顺序插入$T$中,使$P$成为该新序列的子序列。
            此时该新序列为$T$和$P$的公共超序列,其长度为$|T| + |P| - |Q| $。

            所以\[
                |S| = \min (|T| + |P| - |Q|) = |T| + |P| - \max (|Q|) = |T| + |P| - |L|
            \]

            求$|L|$的算法见算法\ref{0330:LCS}。
        \end{solution}
    \end{parts}

    \question 给定规模为$n$的整数数组,如何知道其中是否有\textbf{两个数之和}恰好等于$x$。
    给出算法及相应的时间复杂性。
    \begin{solution}
        记该数组为$A$

        \textsf{解1}\quad
        建立一个可以容纳$n$个整数的哈希表$H$。
        扫描整个数组,到第$i$个数时,计算$r := x - A[i]$。
        若$r$不在哈希表中,记$H[r] = i$;否则有$A[i] + A\left[H[r]\right] = x$。
        哈希表查找和插入的时间复杂度为$O(1)$,所以遍历整个数组的时间复杂度为$O(n)$。

        \begin{algorithm}[H]
            \caption{求补1}\label{0330:Composite1}
            \begin{algorithmic}[1]
                \Require{ $A[1 \dots n]$(被查找的数组) }
                \Ensure{ 是否存在两个数之和恰好等于$x$ }
                \State $H \gets \mathsf{Hashmap}(n)$
                \For {$i \gets 1$ to $n$}
                \State $r \gets x - A[i]$
                \If {$H.contains(r)$}
                \State \Return \textsf{true} \Comment{此时$A[i] + A\left[H[r]\right] = x$}
                \Else
                \State $H[r] \gets i$
                \EndIf
                \EndFor
                \State \Return \textsf{false}
            \end{algorithmic}
        \end{algorithm}
        % \end{solution}

        % \begin{solution}
        \textsf{解2}
        将数组中的所有数分成两个堆,超过$x/2$的所有数插入最大化堆$S$,不足$x/2$的所有数插入最小化堆$I$。
        去两个堆最顶端的数$s,i$的和并与$x$比较:
        \begin{itemize}
            \item 若$s + i = x$,则恰好符合条件
            \item 若$s + i > x$,则移除$S$最顶端的元素并重新调整堆。因为$I$中所有的元素都比$i$大,不可能有一个$i'$使得$i' + s \le i$
            \item 若$s + i < x$,则移除$I$最顶端的元素并重新调整堆。因为$S$中所有的元素都比$s$大,不可能有一个$s'$使得$i + s' \ge i$
        \end{itemize}

        建堆的时间复杂度为$O(n \log n)$,出堆的时间复杂度最大为\[
            a \log a + (n-a) \log (n-a) \le a \log n + (n-a) \log n = n \log n
        \]
        所以该算法的时间复杂度为$O(n\log n)$。

        \begin{algorithm}[H]
            \caption{求补2}\label{0330:Composite2}
            \begin{algorithmic}[1]
                \Require{ $A[1 \dots n]$(被查找的数组) }
                \Ensure { 是否存在两个数之和恰好等于$x$ }
                \State $S \gets MaximizeHeap(), I \gets MinimizeHeap(), halfcnt \gets 0$
                \For {$i \gets 1$ to $n$} \Comment{建堆}
                \If {$2A[i] > x$}
                \State $S.Push(A[i])$
                \ElsIf {$2A[i] < x$}
                \State $I.Push(A[i])$
                \Else \Comment{$2A[i] = x$}
                \State $halfcnt \gets halfcnt + 1$
                \EndIf
                \EndFor
                \If {$halfcnt \ge 2$} \Comment{数组中恰好存在2个或更多$x/2$}
                \State \Return \textsf{true}
                \EndIf
                \While {$\neg S.empty \wedge \neg I.empty$} \Comment{出堆}
                \State $s \gets S.top, i \gets I.top$ \Comment{取堆顶的元素}
                \If {$s+i > x$}
                \State $S.Pop()$
                \ElsIf {$s+i < x$}
                \State $I.Pop()$
                \Else \Comment{$s+i = x$}
                \State \Return \textsf{true}
                \EndIf
                \EndWhile
                \State \Return \textsf{false}
            \end{algorithmic}
        \end{algorithm}
    \end{solution}

    \question 每个螺母需要一个螺栓配套使用,现有$n$个不同尺寸的螺母和相应的$n$个螺栓,如何快速地为为每一个螺母找到对应的螺栓?
    只能将一个螺母与一个螺栓进行匹配尝试,从而知道相互之间的大小关系,
    不能够比较两个螺母的大小,也不能比较两个螺栓的大小。
    给出算法及相应的时间复杂性。
    \begin{solution}
        (类似于快排)

        \textbf{分组方法:}
        随机取一个螺栓(记其大小为$x_0$),与每个螺母都进行一次比较。
        在找到配对的螺母的同时,把比这个螺栓小的螺母和比它大的螺母分成两部分$Y_0, Y_1$。
        有\[
            \forall y_i \in Y_0, y_j \in Y_1,  y_i < x_0 < y_j
        \]
        然后从$Y_0, Y_1$两堆螺母里各选一个螺母($y_0, y_1$),和螺栓匹配,同时将螺栓按小、中、大分成三堆$X_0, X_1, X_2$。
        且有\[
            \forall x_i \in X_0, x_i < y_0  \quad\quad \forall x_j \in X_2, x_j > y_0
        \]
        此时从$X_0$中任选一个螺栓,其配对的螺母必然在$Y_0$中,候选螺母的数量大约减半。

        \textbf{归纳假设:}
        因为螺栓与螺母是一一配对的,即\[ \forall x_i \in X, \exists y_i \in Y : x_i = y_j \]
        所以则若有$
            \forall x_i \in \hat{X}, x_j \in \complement_X^{\hat{X}} : x_i > x_j
        $,$
            \forall y_i \in \hat{Y}, y_j \in \complement_Y^{\hat{Y}} : y_i > y_j
        $
        则必有\[
            \left| \hat{X} \right| \le \left| \hat{Y} \right| \Rightarrow \hat{X} \subseteq \hat{Y}
        \]
        即与$\hat{X}$中的螺栓匹配的螺母全部都在$\hat{Y}$中。
        所以一定可以用$\hat{X}$中的一个螺栓,将$\hat{Y}$分为两组(除非选中的螺栓匹配的螺母是$\hat{Y}$中最小或最大的)。

        \textbf{算法描述:}
        见算法\ref{0330:Pairwise}

        \textbf{复杂性分析:}
        若只考虑拆分一个集合,不考虑拆分另一个集合时从本集合中抽取的元素。

        将一个规模为$n$的集合分为两部分需要进行$n$次比较,且每分一次元素的总个数少1个。
        若要将集合分为$2^k$个部分,需要比较的次数为$\sum_{i = 0}^{k-1} {(n-i)}$,元素总数减少了$\sum_{i=1}^k i$。
        此时,剩余的元素数量有$n - \frac{(1+k)k}{2}$。使拆分结束,有\begin{align*}
            n - \frac{(1+k)k}{2} = 2^k \Rightarrow
            2n & = 2^{k+1} + k(1+k) \ge 2^{k+1} \\
            k  & \le \log{n}
        \end{align*}
        所以总的比较次数为\begin{align*}
            \sum_{i = 0}^{k-1} {(n-i)} = \frac{k\left[n+n-(k-1)\right]}{2}
             & = - \frac{1}{2} k^2 + (n + \frac{1}{2}) k = - \frac{1}{2} k \left[ k - (2 n + 1) \right] \\
             & \le - \frac{1}{2} (\log n)^2 + (n + \frac{1}{2}) (\log n)                                \\
             & = n\log{n} - \frac{1}{2} (\log n)^2 + \frac{1}{2} (\log n) = O(n \log n)
        \end{align*}

        % 此时,剩余的元素数量有$n - \frac{(1+k)k}{2}$。使拆分结束,必然有\begin{align*}
        %     0 < n - \frac{(1+k)k}{2} \le 2^k
        %     \Rightarrow k(k+1) < 2n & \le 2^{k+1} + (1+k)k              \\
        %     k^2 < k(k+1) < 2n       & \le 2^{k+1} + k^2 + k \le 2^{k+2} \\
        %     \frac{1}{2} k^2 < n     & \le 2^{k+1}                       \\
        %     \log n - 1 \le k        & < \sqrt{n}
        % \end{align*}
        % 所以总的比较次数为\begin{align*}
        %     \sum_{i = 0}^{k-1} {(n-i)} = \frac{k\left[n+n-(k-1)\right]}{2}
        %      & = - \frac{1}{2} k^2 + (n + \frac{1}{2}) k                  \\
        %      & = - \frac{1}{2} k \left[ k - 2 (n + \frac{1}{2}) \right]   \\
        %      & = - \frac{1}{2} k \left[ k - (2 n + 1) \right] \\
        % \end{align*}
        % \[
        %     - \frac{1}{2} (\log{n})^2 + (n + \frac{1}{2}) \log{n}
        %       \le T(n) \le - \frac{1}{2} n + (n + \frac{1}{2}) \sqrt{n}
        % \]
        % 所以$T(n) = \Omega(n\log n)$ $T(n) = O(n^{3/2})$
    \end{solution}


    \begin{algorithm}[H]
        \caption{螺栓螺母配对}\label{0330:Pairwise}
        \begin{algorithmic}[1]
            \Require{ $X[1 \dots n]$(螺栓),$Y[1 \dots n]$(螺母) }
            \Ensure { $P = \left\{ (x_i, y_j) \right\}$(配对的螺栓螺母) }
            \State $\mathcal{X} \gets \left\{X\right\}, \mathcal{Y} \gets \left\{ Y \right\}$ \Comment{初始时,螺栓、螺母都只有一组}
            \Repeat
            \State $\hat{X} \gets \mathcal{X}.back, \hat{Y} \gets \mathcal{Y}.back$ \Comment{取最后面的一组(尺寸最大的)}
            \If{$\hat{X}.length \le \hat{Y}.length$}
            \State $x \gets \Call{Select}{\hat{X}}$ \Comment{从$\hat{X}$中随机选一个$x$,并从$\hat{X}$中删除$x$}
            \State $Y_1, Y_2, y \gets \Call{Split}{x, \hat{Y}}$ \Comment{$Y_1$中的螺母更小一些,$Y_2$中的螺母更大一些}
            \State $\mathcal{Y}.PopBack()$
            \State $\mathcal{Y}.PushBack(Y_1)$ if $\neg Y_1.empty$ \Comment{确保$\mathcal{Y}$中靠前的分组中任何一个螺母的尺寸}
            \State $\mathcal{Y}.PushBack(Y_2)$ if $\neg Y_2.empty$ \Comment{严格小于靠后的分组中所有螺母的尺寸}
            \State $P.PushBack((x,y))$ \Comment{配对的}
            \Else
            \State $y \gets \Call{Select}{\hat{Y}}$ \Comment{从$\hat{Y}$中随机选一个$y$,并从$\hat{Y}$中删除$y$}
            \State $X_1, X_2, x \gets \Call{Split}{y, \hat{X}}$
            \State $\mathcal{X}.PopBack()$
            \State $\mathcal{X}.PushBack(X_1)$ if $\neg X_1.empty$
            \State $\mathcal{X}.PushBack(X_2)$ if $\neg X_2.empty$
            \State $P.PushBack((x,y))$
            \EndIf
            \Until {$\neg \mathcal{X}.empty \wedge \neg \mathcal{Y}.empty$} \Comment{$\mathcal{X},\mathcal{Y}$应当同时为空}
            \Statex
            \Procedure{Split}{$pivot, C$}
            \State $S \gets \Phi, I \gets \Phi, peer$
            \While{$\neg C.empty$}
            \State $c \gets C.PopBack()$
            \If {$c > pivot$}
            \State $S.PushBack(c)$
            \ElsIf {$c < pivot$}
            \State $I.PushBack(c)$
            \Else
            \State $peer \gets c$
            \EndIf
            \EndWhile
            \State \Return $I, S, peer$
            \EndProcedure
        \end{algorithmic}
    \end{algorithm}
\end{questions}

\begin{questions}
    \section{0402}\label{sec:0402}

    \question 用分治法找到数组中的最大数和最小数,若数组规模为2的幂,证明需要的比较次数为$\frac{3}{2} n -2$
    \begin{solution}
        易判断初始条件$T(1) = 0, T(2) = 1$,
        且有递推关系
        \[ T(n) = T(2^m) = 2T(2^{m-1}) + 2 \Rightarrow T(2^m) + 2 = 2T(2^{m-1}) + 4 \]
        即
        \[ \frac{T(2^m) + 2}{T(2^{m-1}) + 2} = 2 \]

        所以当$m \ge 1$时,$T(2^m) + 2 = 3 \cdot 2^{m-1}$,
        可得$T(2^m) = \begin{cases}
                3 \cdot 2^{m-1} - 2 & , m \ge 1 \\
                1                   & , m = 0
            \end{cases}
        $

        即
        \[
            T(n) = \begin{cases}
                \frac{3}{2} n - 2 & , n \ge 2 \\
                1                 & , n = 1
            \end{cases}
        \]
    \end{solution}

    \question 对于任意给定的4个1-10之间的整数(可以相同),判断是否可以通过整数四则运算得到24
    \begin{solution}
        % TODO: complete this
        我们将运算中不具有交换律的情况算作两种运算,并规定$x_1 \leq x_2$,使两个数构成有序数对。
        则任一个有序数对的两个数之间可能有$6$种运算,即 \[
            x_1 + x_2 \quad x_1 \cdot x_2 \quad x_1 - x_2 \quad x_2 - x_1 \quad x_1 / x_2 \quad x_2 / x_1
        \]
        记$Q(a,b)$为$a$与$b$进行这$6$种计算得到的所有结果的集合,$|Q(a,b)| \le 6$。

        记$J(A,B,O,n)$为判断是否存在$A$中的某个数能和$B$中的某个数通过$O$中的运算直接得到$n$。
        对于每一种运算,使用类似于算法\ref{0330:Composite1}中的算法,
        首先扫描一遍较小的数组并在哈希表中记录下互补的值,然后扫描另一个数组看是否存在于哈希表中。
        这一部分需要进行$\min\left\{ |A|, |B| \right\}$次计算和$\max\left\{ |A|, |B| \right\}$次哈希表查找。
        所以$J(A,B)$的时间复杂度为$|O|(|A| + |B|)$。

        4个数的结合次序有两种,可表示为后缀表达式
        \begin{itemize}
            \item $(x_1, x_2, op_1, x_3, x_4, op_2, op_3)$:
                  4个数两两一组分为2组共有$\frac{\binom{4}{2} + \binom{2}{2}}{2}=3$种分组方式。
                  所以讨论全部结果需要执行3次\[
                      J\left( Q(x_1,x_2), Q(x_3, x_4), \left\{+, \times, -, \div, \hat{-} , \hat{\div} \right\}, 24 \right)
                  \]
                  总的运算次数为$3 \times 6 \times 12 = 216$
            \item $(x_1, x_2, x_3, x_4, op_1, op_2, op_3)$:
                  此时若$op_2$与$op_3$的运算顺序是可交换的,则与上一种情况相同。
                  \begin{itemize}
                      \item $op_2, op_3$同为加、减法(可通过增减括号后等价)
                      \item $op_2, op_3$同为乘、除法
                  \end{itemize}
                  因此,给定$op_2$后,$op_3$只有3种选择
                  排列方式共有$\binom{4}{2}\binom{2}{1} = 12$种。
                  \[
                      J\left(Q(Q(x_3, x_4), x_2), \left\{x_1\right\}, \left\{+, -, \hat{-}\right\}, 24\right)
                  \]
                  $Q(x_3, x_4)$的运算结果可由上一种情况的运算结果查表得到。
                  运算次数为$12 \times 6^2 \times 3 = 1296$次
        \end{itemize}

        总运算次数为$1512$。
    \end{solution}

    \question 有$n=2^k$位选手参加一项单循环比赛,即每位选手都要与其他$n-1$位选手比赛,且在$n-1$天内每人每天进行一场比赛
    \begin{parts}
        \part 设计算法以得到赛程安排
        \begin{solution}
            赛程安排可表示为表格$M$如下(以$k=2$为例):
            \begin{center}
                \begin{tabular}{c|cccc}
                      & 1        & 2        & 3        & 4        \\
                    \hline
                    1 & $\times$ & 0        & 1        & 2        \\
                    2 & $-$      & $\times$ & 2        & 0        \\
                    3 & $-$      & $-$      & $\times$ & 1        \\
                    4 & $-$      & $-$      & $-$      & $\times$ \\
                \end{tabular}
            \end{center}
            上表中,$m_{1,2} = 0$表示1号选手和2号选手在第0天进行一场比赛。
            由于自己不能和自己比赛,所以对角线上的值是无意义的。
            同时表格中的值应关于对角线对称,所以只需考虑对角线一侧的值(这里只考虑上方)。
            因此只要以某种算法,使用$[0, n-1)$中的整数填写上表,使每行、每列都不存在重复的数即可。

            填写表格方法为:%(算法\ref{0402:FillTable}):
            \begin{itemize}
                \item 第一个数:\[ m_{1,2} = 0 \]
                \item 第一行其他的数: \[ \forall j : 2 < j \le n, m_{1,j} = \left[ 1 + m_{1,j-1}  \bmod (n-1)  \right] \]
                \item 表中其他的数: \[ \forall i,j : i > 1, j > i, m_{i,j} = \left[ 1 + m_{i-1,j-1}  \bmod (n-1)  \right] \]
            \end{itemize}

            \begin{algorithm}[H]
                \caption{比赛安排}\label{0402:FillTable}
                \begin{algorithmic}[1]
                    \Require {$n$(参加比赛的人数)}
                    \Ensure {$M[1\dots n][1\dots n]$(比赛的安排,$i$号选手与$j$号选手在第$M[i][j]$天比赛)}
                    \State $M[1][2] \gets 0$
                    \For {$j \gets 3$ to $n$}
                    \State $M[1][j] \gets M[1][j-1] \bmod (n-1)$
                    \EndFor
                    \For {$i \gets 2$ to $n$}
                    \For {$j \gets i+1$ to $n$}
                    \State $M[i][j] \gets M[i-1][j] \bmod (n-1)$
                    \EndFor
                    \EndFor
                \end{algorithmic}
            \end{algorithm}

            由于第一行中只有$n-1$个数,所以恰能使用$[0, n-1)$中所有的数填满且不重复。
            由于每一列中最多只有$n-1$个数,且有循环递增关系,所以一定能使用$[0, n-1)$中的数填满而不产生重复。
            由于第一行从左到右有循环递增关系,第二行的值为第一行对应的值$+1$,所以行中的递增关系保持,即每一行中也没有重复的数。
            因此由此方法填写的表格满足要求。

            该表格中安排了$\frac{n^2 - n}{2} = 2^{k-1}(n - 1)$,满足每天$k$场比赛,$n-1$天恰好比完。
            算法的时间复杂度为$O(n^2)$
        \end{solution}

        \part 若比赛结果存放在一矩阵中,针对该矩阵设计$O(n \log n)$的算法,为各个选手赋予次序$P_i$,
        使得$P_1$打败$P_2$,$P_2$打败$P_3$,依此类推。
        \begin{solution}
            对于任意两个选手$i,j$,定义若$i$赢了$j$则$i<j$。
            由于每两个人之间都有一场比赛,所以任意两个人之间都可以按上述定义比较大小,该大小关系可以通过查询比赛结果的数组得到。
            使用一种基于比较的排序算法(如快速排序),即可在$O(n\log n)$的时间内排除顺序。
        \end{solution}
    \end{parts}

    \question 数组$A$中包含$n$个互不相同的整数,用$O(n)$的时间找出$A$中最大的$i$个数,并按从大到小的次序输出($i \leq n^{1/2}$)
    \begin{solution}
        首先使用自底向上的方法建最大堆,然后出堆$i$次。

        \begin{itemize}
            \item 自底向上建堆过程中,每一步比较的次数最多是该节点高度的2倍。
                  对于高度为$i$的节点,记其所有子节点的高度和为$H(i)$,则有
                  \begin{align*}
                      H(i) = 2H(i-1) + 1 \xRightarrow{H(0) = 0} H(i) = 2^{i+1} - i - 2
                  \end{align*}
                  而$n$个节点的最大高度为$\log n$,所以这一部分的时间复杂度为\[
                      T_1(n) = 4n - 2 \lceil \log n \rceil - 4
                  \]
            \item 出堆$i$次所需的时间$ T_2(i) \le i \log n $。
        \end{itemize}
        所以整体的时间复杂度\begin{align*}
            T(n, i) = T_1(n) + T_2(i) & \le (4n - 2\lceil \log n \rceil - 4) + i \log n             \\
                                      & \le 4n - 2\lceil \log n \rceil - 4 + \sqrt{n} \log n        \\
                                      & \le 4n - 2\lceil \log n \rceil - 4 + \sqrt{n} \cdot n^{1/2} \\
                                      & = 5n - 2\lceil \log n \rceil - 4 = O(n)
        \end{align*}

    \end{solution}

\end{questions}
