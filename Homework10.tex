\documentclass[answers]{exam}

\usepackage[UTF8]{ctex}
\usepackage{fontspec}

\setCJKmainfont{SourceHanSerifCN-Regular}[
    Path = fonts/ ,
    Extension = .otf ,
    UprightFont = SourceHanSerifCN-Regular,
    BoldFont = SourceHanSerifCN-Bold,
]

\setCJKsansfont{SourceHanSansCN-Regular}[
    Path = fonts/ ,
    Extension = .otf ,
    UprightFont = SourceHanSansCN-Regular,
    BoldFont = SourceHanSansCN-Bold,
]

\usepackage{fullpage}
\usepackage{multicol}
\usepackage{color}
\usepackage{hyperref}
\usepackage{comment}

\usepackage{caption}
\usepackage{subcaption}

\usepackage{amsmath}
\usepackage{amssymb}
\usepackage{amsthm}
\usepackage{bm}
\usepackage{mathtools}

\newtheoremstyle{example}% name
{3pt}% Space above
{3pt}% Space below
{}% Body font
{}% Indent amount (empty = no indent, \parindent = para indent)
{\normalfont}% Thm head font
{}% Punctuation after thm head
{1em}% Space after thm head: " " = normal interword space;
% \newline = linebreak
{\textsf{\thmname{#1}\thmnumber{#2}\thmnote{#3}}}% Thm head spec (can be left empty, meaning `normal')

\theoremstyle{example}
\newtheorem{example}{例}

\renewenvironment{proof}{\noindent\textsf{证明}\quad}{\hfill $\square$\par}
\newtheorem{lemma}{引理}

\usepackage{algorithm}
\usepackage{algpseudocode}

\floatname{algorithm}{算法}
\renewcommand{\algorithmicrequire}{\textsf{输入:}}
\renewcommand{\algorithmicensure}{\textsf{输出:}}

\renewcommand{\solutiontitle}{}

\usepackage{tikz}
\usepackage{minted}


\title{算法与复杂性 \quad 作业十}

\author{516021910528 - SHEN Jiamin}
\date{\today}

\begin{document}

\maketitle

\begin{questions}
    \section{0420}\label{sec:0420}

    \question 1 设计算法判定平面上n个点是否在一条直线上

    2 设P是包围在给定矩形R中的一个简单多边形,q为R中任意一点,设计高效算法寻找连接q和R外部一点的线段,使得该线段与P相交的边的数量最少
    
    

   
% \newpage %%%%%%%%%%%%%%%%%%%%%%%%%%%%%%%%%%%%%%%%%%%%%%%%%%%%%%%%%%%%%%%%%%%%%%%
\section{0423}\label{sec:0423}

1 给定平面上一组点,已知每个点的坐标,求最远点对之间的距离,即点集的直径。(不得穷举,文献查阅,然后用自己的语言进行算法思想的描述,包括时间复杂性分析)

2 给定测度空间中位于同一平面的n个点,已知任意两点之间的距离dij,存储在矩阵D中,求这组点的直径。
该问题的直观解法就是把D扫描一遍,选择其中最大的元素即可。由于是在一个测度空间中,因此dij满足距离的基本要求,即非负性、对称性和三角不等式,我们就可以给出一种时间亚线性的近似算法。算法很简单,由原来确定性算法的检查整个矩阵改为只随机检查D的某一行,这样时间复杂性就由原来的O(n2)减少为O(n),相对于输入规模n2而言,这是一个时间亚线性的算法。那么时间代价减小的同时,证明解不会小于最优值的一半

3在平面上给定一个有n个点的集合S,求S的极大点。极大点的定义:设p1=(x1,y1)和p2=(x2,y2)是平面上的两个点,如果x1x2并且y1y2,则称p2支配p1,记为p_1≺p_2。点集S中的点p为极大点,意味着在S中找不到一个点q,qp并且p≺q,即p不被S中其它点支配。



\end{questions}

\end{document}