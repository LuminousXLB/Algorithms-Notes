\documentclass[answers]{exam}

\usepackage[UTF8]{ctex}
\usepackage{fontspec}

\setCJKmainfont{SourceHanSerifCN-Regular}[
    Path = fonts/ ,
    Extension = .otf ,
    UprightFont = SourceHanSerifCN-Regular,
    BoldFont = SourceHanSerifCN-Bold,
]

\setCJKsansfont{SourceHanSansCN-Regular}[
    Path = fonts/ ,
    Extension = .otf ,
    UprightFont = SourceHanSansCN-Regular,
    BoldFont = SourceHanSansCN-Bold,
]

\usepackage{fullpage}
\usepackage{multicol}
\usepackage{color}
\usepackage{hyperref}
\usepackage{comment}

\usepackage{caption}
\usepackage{subcaption}

\usepackage{amsmath}
\usepackage{amssymb}
\usepackage{amsthm}
\usepackage{bm}
\usepackage{mathtools}

\newtheoremstyle{example}% name
{3pt}% Space above
{3pt}% Space below
{}% Body font
{}% Indent amount (empty = no indent, \parindent = para indent)
{\normalfont}% Thm head font
{}% Punctuation after thm head
{1em}% Space after thm head: " " = normal interword space;
% \newline = linebreak
{\textsf{\thmname{#1}\thmnumber{#2}\thmnote{#3}}}% Thm head spec (can be left empty, meaning `normal')

\theoremstyle{example}
\newtheorem{example}{例}

\renewenvironment{proof}{\noindent\textsf{证明}\quad}{\hfill $\square$\par}
\newtheorem{lemma}{引理}

\usepackage{algorithm}
\usepackage{algpseudocode}

\floatname{algorithm}{算法}
\renewcommand{\algorithmicrequire}{\textsf{输入:}}
\renewcommand{\algorithmicensure}{\textsf{输出:}}

\renewcommand{\solutiontitle}{}

\usepackage{tikz}
\usepackage{minted}


\title{算法与复杂性 \quad 模拟测试}

\author{516021910528 - SHEN Jiamin}
\date{\today}

\begin{document}

\maketitle

\begin{questions}

    \question 已知$n$个矩形,这些矩形的边都平行于坐标轴
    \begin{parts}
        \part 求出这些矩形的交集
        \begin{solution}
        \end{solution}

        \part 求出这些矩形能够覆盖的面积
        \begin{solution}
        \end{solution}
    \end{parts}

    \question 求$n!$包含质因子$p$的数量,例如$6!$含有4个2,2个3和1个5,并给出算法的时间复杂性
    \begin{solution}
    \end{solution}

    \question 设Fibonacci数列的定义为:
    \begin{align*}
        F(1) &=1 \\
        F(2) &=1 \\
        F(n) &=F(n-1)+F(n-2) \quad (n>2)
    \end{align*}
    证明每个大于$2$的整数$n$都可以写成至多$\log n$个Fibonacci数之和,并设计算法对于给定的$n$寻找这样的表示方式
    \begin{solution}
    \end{solution}

\end{questions}

\end{document}