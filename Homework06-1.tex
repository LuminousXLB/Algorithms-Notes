\section{0323}\label{sec:0323}
\begin{questions}

    \question 设计算法将$1$到$n^2$按顺时针方向由内而外填入一个$n \times n$的矩阵

    % 例如$n=2$时,$
    %     \begin{array}{cc}
    %         2 & 3 \\
    %         1 & 4
    %     \end{array}
    % $;$n=5$时,$
    %     \begin{array}{ccccc}
    %         25 & 10 & 11 & 12 & 13 \\
    %         24 & 9  & 2  & 3  & 14 \\
    %         23 & 8  & 1  & 4  & 15 \\
    %         22 & 7  & 6  & 5  & 16 \\
    %         21 & 20 & 19 & 18 & 17
    %     \end{array}
    % $

    \begin{solution}
        记该矩阵为$A[1..n][1..n]$。
        % 当$n$为奇数时,去掉第一列和最后一行后,剩余的元素为$n-1$维的方阵;
        % 当$n$为偶数时,去掉第一行和最后一行后,剩余的元素为$n-1$维的方阵。

        \begin{algorithm}[H]
            \caption{螺旋方阵}
            \begin{algorithmic}[1]
                \Require {
                    空间$A[1..n][1..n]$
                }
                % \Procedure{Main}{}
                \State \Call{SpiralMatrix}{$A, 0, 0, n$}
                % \EndProcedure
                \Statex
                \Procedure{SpiralMatrix}{$M, top, left, size$}
                \Statex
                \Comment{向以$M[top][left]$为$A[0][0]$的$size$维方阵中填入元素}
                \If{$n=1$}
                \State $M[top][left] \gets 1$
                \ElsIf {$[n \bmod 2] = 1$} \Comment{去掉第一列和最后一行后,剩余的元素为$n-1$维的方阵}
                \State \Call{SpiralMatrix}{$A, top, left+1, n-1$} \Comment{以左上角右侧的元素为基准}
                \For {$i \gets 1$ to $n$}
                \State $M[n][i] \gets (n-1)^2 + n - (i - 1)$ \Comment{填充下边}
                \State $M[i][1] \gets n^2 - (i - 1)$ \Comment{填充左边}
                \EndFor
                \Else \Comment{去掉第一行和最后一行后,剩余的元素为$n-1$维的方阵}
                \State \Call{SpiralMatrix}{$A, top+1, left, n-1$} \Comment{以左上角下侧的元素为基准}
                \For {$i \gets 1$ to $n$}
                \State $M[1][i] \gets (n-1)^2 + i$ \Comment{填充上边}
                \State $M[i][n] \gets (n-1)^2 + n + (i - 1)$ \Comment{填充右边}
                \EndFor
                \EndIf
                \EndProcedure
            \end{algorithmic}
        \end{algorithm}

    \end{solution}


    \question 给定整数数组$A[1..n]$,相邻两个元素的值最多相差$1$。
    设$A[1]=x$,$A[n]=y$,并且$x<y$,输入$z$,$x \leq z \leq y$,判断$z$在数组$A$中出现的位置。
    给出算法及时间复杂性(不得穷举)

    \begin{solution}
        相邻两个元素的值最多相差$1$,即有$A[i+1] - A[i] \in \left\{-1, 0, 1\right\}$。
        可以据此类比连续函数上的零点存在性定理,即

        \begin{center}
            若$A[i] \leq z \leq A[j]$且$i < j$,则一定存在一个$m : i \leq m \leq j$使得$A[m] = z$
        \end{center}

        因此可用二分法,有如下算法:
        \begin{algorithm}[H]
            \caption{\quad}
            \begin{algorithmic}[1]
                \Require{整数数组$A[1..n]$, $A$中的一个整数$z$}
                \Ensure{元素$z$在数组$A$中的位置}
                \State $lidx \gets 1, ridx \gets n$
                \State $mid \gets \left\lfloor \frac{lidx+ridx}{2} \right\rfloor$
                \While{$A[mid] \neq z$}
                \If {$A[mid] < z$}
                \State $left \gets mid$ \Comment{更新后仍有$A[left] < z$}
                \Else \Comment{$A[mid] > z$}
                \State $right \gets mid$ \Comment{更新后仍有$A[right] > z$}
                \EndIf
                \State $mid \gets \left\lfloor \frac{lidx+ridx}{2} \right\rfloor$
                \EndWhile
                \State \Return $mid$
            \end{algorithmic}

        \end{algorithm}

        该算法的时间复杂度为$O(\log n)$

    \end{solution}


    \question 假设有$k$个长度为$n$的有序序列,采用下述方法将这些序列合并成一个具有$kn$个元素的有序序列:
    将前两个序列合并,再并入第三个序列,然后是第四个……
    直至所有序列合并。
    \begin{parts}
        \part 用$k$和$n$表示该方法的时间复杂性
        \begin{solution}
            当合并一个长度为$m$和一个长度为$n$的序列时\begin{itemize}
                \item 赋值操作执行了$m+n$次
                \item 比较操作至少执行$\min\left\{m,n\right\}$次,最多执行$2 \min\left\{m,n\right\}$次
            \end{itemize}
            所以此算法合并$k$个长度为$n$的有序序列所需时间为
            \begin{align*}
                T(0, k, n) = T(n, k-1, n) & = 2n + cn + T(2n, k-2, n)                    \\
                                          & = (2+3)n + 2cn + T(3n, k-3, n)               \\
                                          & = \dots                                      \\
                                          & = \sum_{i=2}^{k} {i \cdot n} + (k-1)cn       \\
                                          & = \frac{(k+2)(k-1)}{2} n + (k-1)cn = O(k^2n)
            \end{align*}

        \end{solution}

        \part 能不能得到一个效率更高的方法来合并这$k$个序列?
        \begin{solution}
            首先将第奇数个序列和第偶数个序列合并,形成$\left\lceil \frac{k}{2} \right\rceil$个最大长度为$2n$的序列;
            然后重复上一步骤,直至最终只剩1个序列。

            不妨设$k = 2^m$
            \begin{align*}
                T(k,n) & = \frac{k}{2} T(2,n) + T \left(\frac{k}{2}, 2n\right) = 2^{m-1} (2+c)n + T \left(2^{m-1} , 2n\right) \\
                       & = 2^{m-1} (2+c)n + 2^{m-2} 2(2+c)n + T \left(2^{m-2} , 4n\right)                                     \\
                       & = 2 \cdot 2^{m-1} (2+c)n + T \left(2^{m-2} , 2^2 n\right)                                            \\
                       & = \dots                                                                                              \\
                       & = (m-1) \cdot 2^{m-1} (2+c)n + T \left(2, 2^{m-1} n\right)                                           \\
                       & = m \cdot 2^{m-1} (2+c)n = O(k n \log {k})
            \end{align*}
        \end{solution}
    \end{parts}


    \question 设$A[1..n]$是一个包含$n$个不同数的数组。
    如果在$i<j$的情况下有$A[i]>A[j]$,则$(i,j)$为$A$中的一个逆序对。
    \begin{parts}
        \part 若$A=\left\{2, 3, 8, 6, 1\right\}$,列出其中所有的逆序对
        \begin{solution}
            \[(2,1) \quad (3,1) \quad (8,6) \quad (8,1) \quad (6,1)\]
        \end{solution}
        \part 若数组元素取自$\left\{ 1, 2, \dots ,n \right\}$,那么怎样的数组含有最多的逆序对?它包含多少个逆序对?
        \begin{solution}
            当数组为递减序列时包含有最多的逆序对,此时序列中任取一对数,都能构成逆序对。

            逆序对数为$\frac{n(n-1)}{2}$。
        \end{solution}
        \part 求任意数组$A$中逆序对的数目,并证明算法的时间复杂性为$\Theta(n \log{n})$
        \begin{solution}
            应用分治法,结合归并排序。
            首先将数组等分成左右两个子序列,分别求出两边的逆序对数(并同时进行排序)。
            然后再求出分属于两个子序列的逆序对,即对每一个左序列中的元素,找出右序列中大于该元素的所有元素个数。
            \[ T(n) = 2T(n/2) + (2+c)n = O(n \log n) \]
        \end{solution}
    \end{parts}

    \begin{algorithm}
        \caption{求逆序对数}
        \begin{algorithmic}[1]
            \Require{数组$A[1..n]$}
            \Ensure{$A$中的逆序对数$m$}

            \State $(m, \_) \gets \Call{MergeInversions}{A[1..n]}$
            \Statex

            \Procedure{MergeInversions}{$A[1..n]$}
            \Statex \Comment{输入一个数组$A$,输出一个二元组,包含该数组中的逆序对数及一个有序的该数组}
            \If{$A.size = 1$}
            \State \Return $(0, A)$
            \ElsIf{$A.size = 2$}
            \If{A[1] > A[2]}
            \State \Return $(1, [A[2], A[1]])$
            \Else
            \State \Return $(0, A)$
            \EndIf
            \Else \Comment{分治}
            \State $(nL, L) \gets \Call{MergeInversions}{A\left[1..\left\lfloor n/2 \right\rfloor\right]}$
            \State $(nR, R) \gets \Call{MergeInversions}{A\left[\left\lfloor n/2 \right\rfloor+1 .. n\right]}$
            \State $k \gets 1, cnt \gets 0$ \Comment{归并}
            \While {$\neg L.empty \wedge \neg R.empty$}
            \If{$ L.front \leq R.front $}
            \State $A[k] = L.PopFront()$
            \Else \Comment{$L[1]$大于$R$的所有剩余元素}
            \State $A[k] = R.PopFront()$
            \State $cnt \gets cnt + R.size$ \Comment{$L[1]$与$R$中所有剩余元素都构成逆序对}
            \EndIf
            \State $k \gets k + 1$
            \EndWhile
            \If{$\neg L.empty$} \Comment{填充剩余元素}
            \State $A[k..n] = L$
            \ElsIf{$\neg R.empty$}
            \State $A[k..n] = R$
            \EndIf
            \State \Return $(cnt, A)$
            \EndIf
            \EndProcedure
        \end{algorithmic}
    \end{algorithm}


\end{questions}
