\section{0521}\label{sec:0521}
\begin{questions}
    \question 已知顶点覆盖问题是NP完全的,那么如果所有顶点的度数都是偶数,
    能不能设计出多项式时间的确定性算法?

    \begin{solution}
        在一个任意的图中,考察所有顶点的度数和。
        因为每条边贡献的总度数为$2$,所以所有顶点的度数和为偶数。
        所以图中度数为奇数的顶点一定有偶数个。

        在该图中添加一个新的三角形,$p$为该三角形的一个顶点,
        顶点$p$与图中所有奇数度的顶点$v_i$之间连一条边。
        因为有偶数个奇数度的顶点,所以生成的图$G'$中所有顶点的度数均为偶数。

        为了覆盖新添加的三角形,该三角形中一定有且只有两个顶点存在于$G'$的最小覆盖中。
        因为顶点$p$的度数一定大于三角形中另外两个顶点,
        所以选中顶点$p$一定不会比不选中顶点p的情况覆盖的边少。
        因此这两个顶点中一定包含顶点p。

        因此从$G'$的顶点覆盖中移除这两个点,就是$G$的顶点覆盖。

        因此求参数为$(G,k)$的顶点覆盖问题可以多项式归约到求参数为$(G',k+2)$的顶点覆盖问题。
        因为前者是NP完全的,所以后者也是NP完全的。

    \end{solution}


    \question 证明3-MAX-SAT问题是NP完全问题。

    3-MAX-SAT问题是指对于给定的由$m$个子句构成的合取范式$F$,每个子句恰好有$3$个文字,
    如何对布尔变量$x_1,x_2, \dots ,x_n$赋值,使得$F$中满足的子句尽可能多。

    \begin{solution}

        3-MAX-SAT问题的判定问题形式是$F$中能否有$k$个子句满足。
        3-SAT问题是3-MAX-SAT问题的特例,即$k=m$,由于3-SAT问题是NP完全的,所以3-MAX-SAT问题也是NP完全的。

        % 已知3SAT问题是NP完全问题,如果能将3SAT问题多项式归约到3-MAX-SAT问题,则3-MAX-SAT问题是NP完全问题。

        % 假设存在一个算法$\mathcal{A}$可以解决3-MAX-SAT问题,
        % \begin{itemize}
        %     \item 应用该算法求解$\{x_i\}_n \gets \mathcal{A}(F)$ ,使得$F$中满足的子句数量最大
        %     \item 计算$F(\{x_i\}_n)$,求出$F$中满足的子句的数量为$k$,该步骤的时间复杂度为$O(m)$
        %     \item 因为CNF满足的条件是其所有子句同时为真,所以$F$为真等价于$k=m$,该步骤的时间复杂度为$O(1)$
        % \end{itemize}

        % 因此3SAT问题可以多项式归约到3-MAX-SAT问题。

    \end{solution}

\end{questions}
